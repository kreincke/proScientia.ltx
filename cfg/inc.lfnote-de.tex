% This file originally comes from 'lrt4cs' [(c) 2020 Karsten Reincke,
% https://www.fodina.de/lrt4cs] that is distributed under the terms
% of CC-BY-3.0-DE (= https://creativecommons.org/licenses/by/3.0/)


\footnote{\textbf{Dieser Text wird unter der XYZ Lizenz veröffentlicht.}
Hier können Bedingungen stehen, unter denen Sie Ihren Text weitergeben.
Gute Kandidaten wären z.B, die Creative Commons Lizenzen
\texttt{https://creativecommons.org/}. Traditionell ist auch die Formel:
\emph{Alle Rechte vorbehalten. Die Verwendung von Text und Bildern, auch
auszugsweise, bedürfen der schriftlichen Zustimmung.}. Aber wie auch immer: da \textit[lrt4cs] unter der Lizenz \texttt{CC BY 3.0 DE} veröffentlicht worden ist, müssen Sie auf dessen Verwendung hinweisen. Ein entsprechender lize-erfüllender Satz könnte z.B. dieser sein:
\newline
{\small \itshape Erarbeitet mit dem Paket \texttt{\textbf{lrt4cs}} \copyright{} 2020 K. Reincke (\href{https://fodina.de/lrt4cs}{https://fodina.de/lrt4cs}), das CC-BY-3.0-DE ( \href{https://creativecommons.org/licenses/by/3.0/}{https://creativecommons.org/licenses/by/3.0/}) lizenziert wurde.}}
