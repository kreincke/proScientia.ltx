% This file is part of proScientia.ltx
% (c) 2022 Karsten Reincke (https://github.com/kreincke/proScientia.ltx)
% It is distributed under the terms of the creative commons license
% CC-BY-4.0 (= https://creativecommons.org/licenses/by/4.0/)

\footnote{\textbf{Dieser Text wird unter der XYZ Lizenz veröffentlicht.}
Hier können Ihre Bedingungen stehen, unter denen Sie Ihren Text weitergeben.
Gute Kandidaten wären z.B. die Creative Commons Lizenzen
\href{https://creativecommons.org/}{https://creativecommons.org/}. Traditionell ist auch die Formel: \emph{Alle Rechte vorbehalten. Die Verwendung von Text und Bildern, auch auszugsweise, bedarf der schriftlichen Zustimmung}. \newline
Da Ihre Arbeit auf dem Templatesystem \textit{proScientia.ltx} aufbaut und da Sie das unter den Bedinungen der \texttt{CC BY 4.0} Lizenz erhalten haben, müssen Sie auf dessen Verwendung hinweisen. Eine lizenzerfüllende Notiz könnte sein:
\newline
\textit{Erstellt auf der Basis des CC-BY-4.0 lizenzierten Tools \texttt{proScientia} von K. Reincke \copyright{} 2022 [
Repository \href{https://github.com/kreincke/proScientia.ltx}{https://github.com/kreincke/proScientia.ltx} ,
Lizenztext \href{https://creativecommons.org/licenses/by/4.0/}{https://creativecommons.org/licenses/by/4.0/} ]
}}
