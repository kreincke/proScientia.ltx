% This file is part of proScientia.ltx
% (c) 2022 Karsten Reincke (https://github.com/kreincke/proScientia.ltx)
% It is distributed under the terms of the creative commons license
% CC-BY-4.0 (= https://creativecommons.org/licenses/by/4.0/)

\documentclass[
  DIV=calc,i
  BCOR=5mm,
  11pt,
  headings=small,
  oneside,
  abstract=true,
  toc=bib,
  english,ngerman]{scrartcl}

%%% (1) general configurations %%%
\usepackage[utf8]{inputenc}
\usepackage{a4}

%%% (2) language specific configurations %%%
\usepackage[english,ngerman]{babel}
%\selectlanguage{ngerman}

\usepackage[
  backend=biber,
  style=authortitle-dw,
  %style=authortitle,
  sortlocale=auto,
  %citestyle=
]{biblatex}
\input{../cfg/inc.cfg-biber-de.tex}

\addbibresource{../bib/lit.test.bib}

% package for improving the grey value and the line feed handling
\usepackage{microtype}

%language specific quoting signs
\usepackage[
  style=german,
  autostyle=true,
]{csquotes}

% language specific hyphenation
\input{../cfg/inc.babelhyphenations.tex}

%%% (3) layout page configuration %%%

% select the visible parts of a page
% S.31: { plain|empty|headings|myheadings }
%\pagestyle{myheadings}
\pagestyle{headings}

% select the wished style of page-numbering
% S.32: { arabic,roman,Roman,alph,Alph }
\pagenumbering{arabic}
\setcounter{page}{1}

% select the wished distances using the general setlength order:
% S.34 { baselineskip| parskip | parindent }
% - general no indent for paragraphs
\setlength{\parindent}{0pt}
\setlength{\parskip}{1.2ex plus 0.2ex minus 0.2ex}


%%% (4) general package activation %%%
%\usepackage{utopia}
%\usepackage{courier}
%\usepackage{avant}
\usepackage[dvips]{epsfig}

% graphic
\usepackage{graphicx,color}
\usepackage{array}
\usepackage{shadow}
\usepackage{fancybox}

\usepackage{tikz}
\usetikzlibrary{arrows}
\usetikzlibrary{shapes,snakes}
\usetikzlibrary{positioning}
\usetikzlibrary{decorations.text}
\usetikzlibrary{trees}
\usetikzlibrary{matrix}

\usepackage{amsmath}
\usepackage{amsfonts}
\usepackage{amssymb}
\usepackage{wasysym}
\usepackage{chngcntr}


%- start(footnote-configuration)

\deffootnote[1.5em]{1.5em}{1.5em}{\textsuperscript{\thefootnotemark)\ }}

% if document class: count footnotes from start to end
%\counterwithout{footnote}{chapter}
%- end(footnote-configuration)

% package for macking tables with broken lines
\usepackage{multirow}

%for using label as nameref
\usepackage{nameref}

%integrate nomenclature
\input{../cfg/inc.cfg-ncl-de.tex}

% depth of contents
\setcounter{secnumdepth}{5}
\setcounter{tocdepth}{5}

% Hyperlinks
\usepackage{hyperref}
\hypersetup{bookmarks=true,breaklinks=true,colorlinks=true,citecolor=blue,draft=false}

\begin{document}

%% use all entries of the bliography
\nocite{*}

%%-- start(titlepage)
\titlehead{Bib\LaTeX}
\subject{Release 1.0}
\title{Verifikation der deutschen Konfiguration}
\subtitle{Kurztest anhand einer abstrakten Bibliographie}
\author{Autor\footnote{
\textit{Erstellt auf der Basis des CC-BY-4.0 lizenzierten Templatesystems \texttt{proScientia} von Karsten Reincke \copyright{} 2022 [
Repository: \href{https://github.com/kreincke/proScientia.ltx}{https://github.com/kreincke/proScientia.ltx} ,
Lizenztext: \href{https://creativecommons.org/licenses/by/4.0/}{https://creativecommons.org/licenses/by/4.0/} ]}}
}

%thanks entry cannot be combined with license footnote
%\thanks{den Autoren von KOMA-Script und denen von Jurabib}

\maketitle
%%-- end(titlepage)

\footnotesize
\tableofcontents

\normalsize

\section{Test}
\begin{description}
  \item[\mars]: Complete\footcite[vgl.][15]{ICOLLa}, ders. ebda.\footcite[vgl.][15]{ICOLLa}, ders. a.a.O.\footcite[vgl.][23]{ICOLLa}, break\footcite[vgl.][15]{BR2020a}, ShortTitle\footcite[vgl.][15]{ICOLLa}

  \item[\mars]: Complete\footcite[vgl.][15]{IPROCa}, ders. ebda.\footcite[vgl.][15]{IPROCa}, ders. a.a.O.\footcite[vgl.][23]{IPROCa}, break\footcite[vgl.][15]{BR2020a}, ShortTitle\footcite[vgl.][15]{IPROCa}
\end{description}

% This file is part of proscientia.ltx
% (c) 2022 Karsten Reincke (https://github.com/kreincke/proscientia.ltx)
% It is distributed under the terms of the creative commons license
% CC-BY-4.0 (= https://creativecommons.org/licenses/by/4.0/)


% specific abbreviations
\abbr[utb]{UTB}{Uni-Taschenbuch}
\abbr[stw]{stw}{suhrkamp taschenbuch wissenschaft}

% general abbreviations
\abbr[vgl]{vgl.}{vergleiche}
\abbr[aaO]{a.a.O.}{am angegebenen Ort}
\abbr[ebda]{ebda.}{ebenda}
% \abbr[id]{id.}{idem = latin for 'the same', be it a man, woman or a group\ldots}
% \abbr[ibid]{ibid.}{ibidem = latin for 'at the same place'}
\abbr[ifross]{ifross}{Institut für Rechtsfragen der Freien und Open Source
Software}
% \abbr[lc]{l.c.}{loco citato = latin for 'in the place cited'}
\abbr[wp]{wp.}{webpage = Webdokument ohne innere Seitennummerierung}

% This file originally comes from 'lrt4cs' [(c) 2020 Karsten Reincke,
% https://www.fodina.de/lrt4cs] that is distributed under the terms
% of CC-BY-3.0-DE (= https://creativecommons.org/licenses/by/3.0/)

\abbr[afda]{AfdA}{Anzeiger für deutsches Altertum}
\abbr[zfda]{ZfdA}{Zeitschrift für deutsches Altertum und deutsche Literatur [ISSN: 00442518]}
\abbr[zfaw]{}{Zeitschrift für Allgemeine Wissenschaftstheorie / Journal for General Philosophy of Science [ISSN: 0044-2216]}

\printnomenclature
\printbibliography

\end{document}
