% This file is part of proScientia.ltx
% (c) 2022 Karsten Reincke (https://github.com/kreincke/proScientia.ltx)
% It is distributed under the terms of the creative commons license
% CC-BY-4.0 (= https://creativecommons.org/licenses/by/4.0/)

\documentclass[
  DIV=calc,
  BCOR=5mm,
  11pt,
  headings=small,
  oneside,
  abstract=true,
  toc=bib,
  english,ngerman]{scrartcl}

%%% (1) general configurations %%%
\usepackage[utf8]{inputenc}
\usepackage{a4}
\usepackage[english,ngerman]{babel}

\usepackage[
  backend=biber,
  style=authortitle-dw,
  sortlocale=auto,
]{biblatex}
\input{../cfg/inc.cfg-biber-de.tex}

\addbibresource{../bib/lit.verify.bib}
\addbibresource{../bib/lit.main.bib}

% package for improving the grey value and the line feed handling
\usepackage{microtype}

%language specific quoting signs
\usepackage[
  style=german,
  autostyle=true,
]{csquotes}

% language specific hyphenation
\input{../cfg/inc.babelhyphenations.tex}

%%% (3) layout page configuration %%%

% select the visible parts of a page
% S.31: { plain|empty|headings|myheadings }
\pagestyle{headings}

% select the wished style of page-numbering
% S.32: { arabic,roman,Roman,alph,Alph }
\pagenumbering{arabic}
\setcounter{page}{1}

% select the wished distances using the general setlength order:
% S.34 { baselineskip| parskip | parindent }
% - general no indent for paragraphs
\setlength{\parindent}{0pt}
\setlength{\parskip}{1.2ex plus 0.2ex minus 0.2ex}


%- start(footnote-configuration)

\deffootnote[1.5em]{1.5em}{1.5em}{\textsuperscript{\thefootnotemark)\ }}

%for using label as nameref
\usepackage{nameref}

%integrate nomenclature
\input{../cfg/inc.cfg-ncl-de.tex}

% Hyperlinks
\usepackage{hyperref}
\hypersetup{bookmarks=true,breaklinks=true,colorlinks=true,citecolor=blue,draft=false}

\begin{document}

%% use all entries of the bliography
\nocite{*}

%%-- start(titlepage)
\titlehead{Bib\LaTeX}
\subject{Release 1.0}
\title{Test bibliographischer Angaben\footnote{
\textit{Erstellt auf der Basis des CC-BY-4.0 lizenzierten Templatesystems \texttt{proScientia} von Karsten Reincke \copyright{} 2022 [
Repository: \href{https://github.com/kreincke/proScientia.ltx}{https://github.com/kreincke/proScientia.ltx} ,
Lizenztext: \href{https://creativecommons.org/licenses/by/4.0/}{https://creativecommons.org/licenses/by/4.0/} ]}}
}


\maketitle
%%-- end(titlepage)
\begin{abstract}
\noindent \itshape
Ein bibliographischer Eintrag in \texttt{lit.main.bib} wird fallweise ausgetestet:
\end{abstract}


%%%%%%%%%%%%%%%%%%%%%%%%%%%%%%%%%%%%%%%%%%%%%%%%%%%%%%%%%%%%%%%%%%%%%%%%
% Replace KantKdrV1974 by the BibtexKey of the dataset you want to test %
%%%%%%%%%%%%%%%%%%%%%%%%%%%%%%%%%%%%%%%%%%%%%%%%%%%%%%%%%%%%%%%%%%%%%%%%

\section{Kontextsensitiver Test}

\begin{itemize}
  \item \enquote{Alle Daten} = Initialzitat\footcite[vgl.][123]{KantKdrV1974}
  \item \enquote{ders. ebda} = dasselbe Werk, dieselbe Seite\footcite[vgl.][123]{KantKdrV1974}
  \item \enquote{ders. a.a.O. p.} = dasselbe Werk, andere Seit\footcite[vgl.][125f]{KantKdrV1974}
  \item \enquote{ders. + Daten} = derselbe Autor, anderes Werk\footcite[vgl.][321]{KantKdpV1974} (Simulation\footcite[vgl.][42]{KantKdU1974} )
  \item \enquote{Kurztitel} = erstes Werk, erste Seite, anderer  Kontext\footcite[vgl.][123]{KantKdrV1974}
\end{itemize}

\input{../bib/ncl.abbrevs-de.tex}
% This file originally comes from 'lrt4cs' [(c) 2020 Karsten Reincke,
% https://www.fodina.de/lrt4cs] that is distributed under the terms
% of CC-BY-3.0-DE (= https://creativecommons.org/licenses/by/3.0/)

\abbr[afda]{AfdA}{Anzeiger für deutsches Altertum}
\abbr[zfda]{ZfdA}{Zeitschrift für deutsches Altertum und deutsche Literatur [ISSN: 00442518]}
\abbr[zfaw]{}{Zeitschrift für Allgemeine Wissenschaftstheorie / Journal for General Philosophy of Science [ISSN: 0044-2216]}

\printnomenclature
\printbibliography

\end{document}
