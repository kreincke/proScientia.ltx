% This file is part proScientia.ltx/humanities.
% (c) 2022 K. Reincke (https://github.com/kreincke/proScientia.ltx/humanities)
% Unlike the other files of proScientia, all files under proScientia/humanities
% are licensed under the terms of the creative commons license CC-BY-SA-4.0
% (= https://creativecommons.org/licenses/by-sa/4.0/)

\documentclass[
  DIV=calc,
  BCOR=5mm,
  11pt,
  headings=small,
  oneside,
  abstract=true,
  toc=bib,
  english,ngerman]{scrartcl}

% language specific attributes
\usepackage[utf8]{inputenc}
\usepackage{a4}
\usepackage[english,ngerman]{babel}
% This file is part proScientia.ltx/humanities.
% (c) 2022 K. Reincke (https://github.com/kreincke/proScientia.ltx/humanities)
% Unlike the other files of proScientia, all files under proScientia/humanities
% are licensed under the terms of the creative commons license CC-BY-SA-4.0
% (= https://creativecommons.org/licenses/by-sa/4.0/)

\hyphenation{ pro-scien-tia there-fo-re}

\babelhyphenation{ my-keds wor-tmi-tfalsch-entr-ennu-ngen}

\usepackage[
  style=german,
  autostyle=true,
]{csquotes}

% quoting with foot- anmd endnotes
\usepackage[
  backend=biber,
  style=authortitle-dw,
  sortlocale=auto,
]{biblatex}
\ProvidesFile{biblatex.cfg}
% This file is part proScientia.ltx/humanities.
% (c) 2022 K. Reincke (https://github.com/kreincke/proScientia.ltx/humanities)
% Unlike the other files of proScientia, all files under proScientia/humanities
% are licensed under the terms of the creative commons license CC-BY-SA-4.0
% (= https://creativecommons.org/licenses/by-sa/4.0/)



\ExecuteBibliographyOptions{
  %%%%%%%%%%%%%%%%%%%%%%%%%%%%%%%%%%%%%%%%%%%%
  % configurations offered by biblatex itself
  % ------------------------------------------
  maxnames=5,  % Truncate author list after 5 authors ...
  minnames=3,  % ... But display at least 3 authors
  autocite=inline,
  hyperref=true,  % Use hyperref package (should be automatically detected, though)
  backref=true,  % Back references from bibliography page to each encounter
  backrefstyle=two,  % Combine back refs if on two consecutive pages
  isbn=true,  % (Dont) print ISBN, ISSN numbers
  autolang=hyphen,
  % track 'reused' csquotes
  citetracker=constrict, %
  loccittracker=constrict, % discriminate different pages (a.a.O) versues same page (ebda)
  opcittracker=constrict,
  idemtracker=constrict,
  ibidtracker=constrict,
  pagetracker=true,
  %%%%%%%%%%%%%%%%%%%%%%%%%%%%%%%%%%%%%%%%%%%%
  % configurations offered by authortitle-dw
  % ------------------------------------------
  annotation =true,
  namefont=normal,
  firstnamefont=normal,
  idemfont=italic,
  ibidemfont=italic,
  idembib=true, % cluster the books of the same author in bib
  idembibformat=idem, % indicate the same author by ders/dies.
  editorstring=brackets, % parens=(Hrsg.) | brackets=[Hrsg.] | normal = , Hrsg.
  nopublisher=false, % insert publisher into bib data
  editionstring=true, % allow strings in the edition field
  % ZNAME VOL (YEAR) Nr. YOURNALNUMBER
  journalnumber=afteryear, %
  %journumstring=h.
  series=afteryear,
  seriesformat=parens,
  addyear=true, %insert year after titel in 'shorttitle' hints => no year in shorttitle
  firstfull=true, % frist quote complete
  edstringincitations=false, %editor and translator only in the first
  citepages=separate, % vollzitat erst mit seiten, dann 'hier: S. 12'
}
% Put your definitions here.

% refine seriesformat by adding a prefix inside of the parens
\renewcommand*{\seriespunct}{=\addspace}

% by default Biblatex-dw uses Autor1/Autor2 in cites
% these redefinitions overwrite that behaviour by
% duplicating the style used for the bibliography
% (S. p. 35 in the German biblatex-dw handbook )
\renewcommand*{\citemultinamedelim}{\addcomma\space}
\renewcommand*{\citefinalnamedelim}{%
\ifnum\value{liststop}>2 \finalandcomma\fi
\addspace\bibstring{and}\space}%
\renewcommand*{\citerevsdnamedelim}{\addspace}

% biblatex-dw does not print 'ders.' / 'dies.' in a row of cites quoting
% the same book. Additionally, it does not know the diffrence between the
% same page of of the same work and a different page of the same work
% quoted before.
%
% As soon as biblatex 3.15 is offered by UBUNTU
% use \bibncpstring[\mkibid]{ibidem} instaed of inserting the
% German string literally as this hack does:
\renewbibmacro*{cite:ibid}{%
{\ifthenelse{\ifloccit}
  {\printtext[bibhyperref]{\usebibmacro{cite:idem} \mkibid{ebda}}%
   \global\booltrue{cbx:loccit}}
  {\printtext[bibhyperref]{\usebibmacro{cite:idem} \mkibid{a.a.O, }}}
}
}%
\DefineBibliographyStrings{english}{%
  urlseen = {reviewed},
}
\DefineBibliographyStrings{german}{%
  urlseen = {heruntergeladen am},
}
\endinput

\usepackage{enotez}
\addbibresource{humanities.bib}
\deffootnote[1.5em]{1.5em}{1.5em}{\textsuperscript{\thefootnotemark)\ }}

% for page design
\pagestyle{headings}
\pagenumbering{arabic}
\setcounter{page}{1}
\setlength{\parindent}{0pt}
\setlength{\parskip}{1.2ex plus 0.2ex minus 0.2ex}

% chapter design
\setcounter{secnumdepth}{5}
\setcounter{tocdepth}{5}

% special signs
\usepackage{amsfonts}

% make tables with broken lines
\usepackage{multirow}
\usepackage{chngcntr}
\usepackage{longtable}
\usepackage{mwe}

%for using label as nameref
\usepackage{nameref}

%integrate nomenclature
% This file originally comes from 'lrt4cs' [(c) 2020 Karsten Reincke,
% https://www.fodina.de/lrt4cs] that is distributed under the terms
% of CC-BY-3.0-DE (= https://creativecommons.org/licenses/by/3.0/)

\usepackage[intoc]{nomencl}
\let\abbr\nomenclature
% Deutsche Überschrift
%\renewcommand{\nomname}{Abbreviations}
\renewcommand{\nomname}{Abkürzungen}

\setlength{\nomlabelwidth}{.20\hsize}
\renewcommand{\nomlabel}[1]{#1 \dotfill}
% reduce the line distance
\setlength{\nomitemsep}{-\parsep}
\makenomenclature


% Hyperlinks
\usepackage{hyperref}
\hypersetup{bookmarks=true,breaklinks=true,colorlinks=true,citecolor=blue,draft=false}
\newcommand{\lnka}[1]{\href{#1}{\texttt{#1}}}
\newcommand{\lnkb}[2]{\href{#1}{\texttt{#1} (Referenzdownload: #2)}}
\newcommand{\lnkc}[2]{\href{#1}{\texttt{#2}}}

% word design
\newcommand{\ttt}[1]{\texttt{#1}}
\newcommand{\tit}[1]{\textit{#1}}
\newcommand{\tbf}[1]{\textbf{#1}}

\begin{document}

\nocite{*}

%%-- start(titlepage)
\titlehead{Das proScientia-Manifest unter proScientia.ltx/humanities/}
\subject{Wissenschaftliche Texte als Dienst an der Leserin}
\title{Über hilfreiche Anmerkungsapparate}
\subtitle{Geisteswissenschaft mit \LaTeX, \emph{JabRef}, \emph{biber} und Bib\LaTeX}
\author{Karsten Reincke% This file is part proscientia.ltx/humanities.
% (c) 2022 K. Reincke (https://github.com/kreincke/proscientia.ltx/humanities)
% Unlike the other files of proscientia, all files under proscientia/humanities
% are licensed under the terms of the creative commons license CC-BY-SA-4.0
% (= https://creativecommons.org/licenses/by-sa/4.0/)

\footnote{Diesen Text veröffentliche ich unter den Bedingungen der \lnkc{https://creativecommons.org/licenses/by-sa/4.0}{CC-BY-SA 4.0} Lizenz. Technisch abgeleitet ist er vom Framework \tit{proscientia.ltx}: \copyright{} 2022 Karsten Reincke (= \lnka{github.com/kreincke/proscientia.ltx}, lizenziert unter der \lnkc{https://creativecommons.org/licenses/by/4.0}{CC-BY 4.0} Lizenz.)}
}

\maketitle
%%-- end(titlepage)

\begin{abstract}
\noindent \itshape
Der Umgang mit Quellennachweisen hat immer auch rezeptive Auswirkungen: wenn er gut ist, erleichtert er das lernende Lesen. Das gilt besonders für den (alt)philologisch/pholosophischen Anmerkungsapparat. Dieser Artikel beschreibt und zeigt an sich selbst, wozu und wie so etwas mit \LaTeX, Bib\LaTeX\ und \emph{Biber} erzeugt wird\footnote{Um es noch schärfer zu sagen: Quellenangaben dienen mir of nur dazu, im Text angesprochene Formen vorzuführen, und nicht dazu, Passagen des Textes zu belegen.}: Zuerst erklärt er, was ich mir wünsche, dann, wozu das gut ist, und schließlich, wie das technisch geht.
\end{abstract}

\footnotesize
\tableofcontents
\normalsize


\section{Worum es geht.}

\begin{quote}
  \tit{In diesem Quasitutorial\footnote{Dieser Artikel ist im eigentlichen Sinne keine Anleitung, wie Sie -- liebe Leserin -- die (alt)\-philologisch philosophische Nachweis- und Zitiermethode in und mit \LaTeX\ nutzen. Vielmehr ist es eine selbstreferentielle Demo: Sie können an und in dem Text sehen, was gemeint ist. Sie können in den Quelldateien nachsehen, wie es gemacht ist. (\cite[s. dazu][]{Reincke2022a}). Allerdings fange ich nicht bei Null an: Wie Sie \LaTeX\ nutzen, woher Sie es bekommen, wie Sie es installieren und wie Sie einen \LaTeX-Text schreiben, sollten Sie schon wissen.} nutze ich das \tbf{generische Femininum}: Nach 1000 Jahren im Vordergrund der deutschen Sprache steht es den Männern gut an, Frauen auch dort vorzulassen.\footnote{Es ist klar: Um Männer ihrerseits nicht zu benachteiligen, werden Frauen kaum fordern, generell das \emph{generische Femininum} zu nutzen. Aber wir Männer können das von uns aus anbieten. Am besten wäre es überhaupt, Frauen nutzten konsequent das \emph{generische Maskulinum} und meinten Frauen mit, Männer dagegen nutzten immer das \emph{generische Femininum} und meinten Männer mit.} Darum gilt hier: \tbf{Männer sind mitgemeint}.}
\end{quote}

Die (alt)philologisch/philosophische Nachweis- und Zitiermethode nahm mich von Anfang an gefangen, selbst wenn ich diese Liebe zu Fuß- und Endnoten später nicht immer ausleben konnte.

Trotzdem wünschte ich mir ziemlich bald, sie auch in und mit \LaTeX\ umsetzen zu können - obwohl ich wusste, dass \LaTeX\ eben nicht aus der europäischen Geisteswissenschaft heraus entstanden war, sondern aus der computerisierten Mathematik und der anglo-amerikanischen Schreibtradition, wie sie im \textit{Handbook for Writers of Research Papers}\footcite[vgl.][]{ModLanAss2009a} spezifiziert ist.

Meine Leidenschaft für die europäische Tradition ließ mich über die Jahre sogar immer wieder einmal schon existierende Stil- und Bibliotheksdateien ausreizen. Ich versuchte \emph{natib}\footcite[vgl.][]{Daly2000a} anzupassen, dann \emph{custom-bib}\footcite[vgl.][]{Daly2007a} und zuletzt \emph{Jurabib}\footcite[vgl.][]{Berger2004a}. Damit näherte ich mich meinem Ideal wenigstens so weit an, dass ich mit  \emph{mycsrf}\footcite[vgl.][]{Reincke2021a} auch eine auf \LaTeX\, \emph{make} und \emph{bash} gründende \emph{IDE} zusammenstellen konnte. Nur perfekt war diese Methode nie. Doch nun bin ich auf \emph{Dominik Waßenhoven} und seine \emph{biblatex-dw}-Lösung\footnote{\cite[vgl.][]{Waßenhoven2016a}. Dieses Paket enthält auch das von demselben Autor geschriebene Handbuch, das aus dem {ctan}-Package-Frontend heraus ausgerufen werden kann. Links in ctan-Pakete werden \emph{ctan} allderings standardmäßig auf Spiegelserver umgeleitet. Es macht also keinen Sinn, einen in Wirklichkeit ja zufällig gewählten Server über das Feld \ttt{url} im bibliographischen Datensatz als \textbf{die} \emph{biblatex-dw-Handbuch-Quelle} 'anzugeben'. Darum verwende ich die bibliographische Angaben für das Paket im Allgemeinen zugleich auch als die für das Handbuch im Besonderen.} gestoßen.

Eigentlich waren meine Wünsche von Anfang an einfach: Ich mag keine Zitate, die innerhalb des Textes durch esoterische Nummernblöcke[42] oder kryptische Bib\TeX-Keys[Daly2000a] belegt werden. Das sind Stolpersteine im Lesefluss. Zudem liebe ich den Subtext, den ein Anmerkungsapparat aufspannt: ich schätze es, wenn mich Autorinnen über die konzentrierte Argumentation ihrer Haupttexte hinaus lustvoll auf Nebenwegen durch die mäandernde Forschungsgeschichte führen. Dieses schwärmenden Hin und Her zwischen Text und Fuß-\footnote{Hier kann ich -- je nach Lust, Laune oder Notwendigkeit -- mit einem Blick nach unten überprüfen oder komplettieren, was mir notwendig oder sinnvoll erscheint. Anmerkungen bringen mir die Forschungsgeschichte nahe, ganz nebenbei - wenn die Nachweise sprechende Belege sind.} oder Endnoten\endnote{Obwohl sie dieselben Optionen wie Fußnoten bieten, zwingen sie mich jedoch zu blättern - was letztlich doch auch stört.} möchte ich genießen.

Doch es bedarf bestimmter Optionen, wenn die Forschungsgeschichte in einem Anmerkungsapparat wirklich leserinnenfreundlich aufgearbeitet werden soll. So wird eine Autorin in einer Fußnote\footnote{wenn sie z.B. nachweisen wollte, dass der \emph{Online-Duden} das Wort \enquote{Dippelschisser} 02/2022 (noch) nicht kennt (\cite[vgl.][]{DudenOnline2022a}), wohl aber der \emph{offene Thesaurus} [\cite[vgl.][]{OpenThesaurus2022a}] oder die \emph{Site der deutschen Synonyme} (\cite[vgl.][]{DeutscheSynonyme2022a})} oder Endnote\endnote{Nur der Form halber dieselbe Anwendung noch einmal: der \emph{Online-Duden} kennt das Wort \enquote{Dippelschisser} 02/2022 (noch) nicht (\cite[vgl.][]{DudenOnline2022a}), wohl aber der \emph{offene Thesaurus} [\cite[vgl.][]{OpenThesaurus2022a}] oder die \emph{Site der deutschen Synonyme} (\cite[vgl.][]{DeutscheSynonyme2022a}).} gelegentlich mehrere Belege unterbringen wollen. Und die Quellennachweise müssen sprechend aufgeschlüsselt sein, bei der ersten Erwähnung bibliographisch vollständig\footnote{\emph{wie hier}: \cite[vgl.][S. 195ff]{RueStaFra1980a}}, samt aller Autorinnen, allen Titelfeinheiten und der Auflage. Als Leserin möchte ich jedenfalls auch auf Übersetzerinnen und Reihen hingewiesen werden, möchte editorische Sonderfälle erkennen und die ISBN oder ISSN erfahren\footcite[\emph{wie hier}: vgl.][]{Covey2006a} - auf dass mir das Wiederfinden der zitierten Werke möglichst leicht gemacht werde.

Erst wenn im Laufe der Argumentation erneut auf dieselbe Quelle zurückgegriffen wird\footcite[\emph{wie hier}: vgl. dazu][194]{RueStaFra1980a}, reicht mir ein verkürzter Beleg\footcite[\emph{wie hier}: vgl.][]{Covey2006a}, der mich zwanglos auf die mnemotechnisch richtige Bahn bringt, ohne mir eine verklausulierte Geheimsprache aufzunötigen: denn \emph{kurz} meint schließlich nicht \emph{esoterisch}\footnote{Es funktioniert also: Den \emph{Rückriem/Stary/Franck} habe ich gerade bibliographisch ebenso ausführlich ausgewiesen, wie den \emph{Covey}. Anschließend habe ich erneut auf beide Quellen verwiesen, nun aber über die Kurzform \emph{Autorin: Titel. (Jahr)}. Die in Bib\LaTeX\ und \emph{biblatex-dw} eingebundenen Features erledigen das automatisch.}.

Wenn allerdings Autorin und Leserin sich im Haupttext über eine längere Strecke auf eine verfeinerte Analyse fremder Gedankengängen einlassen\footcite[\emph{wie hier}: vgl. etwa][32]{Allen2001a} und dabei \footcite[\emph{wie hier}: vgl.][139]{Allen2001a} verschiedene Passagen\footcite[\emph{wie hier}: vgl.][139]{Allen2001a} desselben Werkes begutachten, dann wäre auch der wiederholte Kurznachweis unnötig geschwätzig. Eigentlich reichten ja kürzere Indikatoren: Nehmen wir an, wir zitierten zuerst eine Stelle aus der Kritik der Urteilskraft\footcite[vgl.][9]{KantKdU1974}, dann eine aus der Kritik der reinen Vernunft\footcite[vgl.][45]{KantKdV1974} und dann zwei andere Passagen daraus\footcite[vgl.][69]{KantKdV1974}, nämlich aus der \emph{Transzendentale Ästhetik}\footcite[vgl.][69]{KantKdV1974}. Dann müßte unser Anmerkungsapparat zunächst die beiden ersten Belege komplett auflisten, wobei er schon den zweiten mit einem \emph{ders.} an den ersten anbinden könnte. Und er sollte als nächstes ein \emph{ders., a.a.O. + neue Seite} enthalten, gefolgt von einem
\emph{ders., ebda}. Gingen wir nun zurück auf die \emph{Kritik der Urteilskraft}\footcite[vgl.][9]{KantKdU1974}, dürfte unser Apparat nicht mehr mit dem Kürzel \emph{ders., a.a.O} arbeiten, sondern sollte mit \emph{ders.: (Kurz)Titel} + Jahr einen neuen Bezugspunkt setzen. Und dasselbe sollte auch mit den Büchern einer Autorin möglich sein, allerdings genderkonform: Zuerst das eine Werk\footcite[\emph{wie hier:} vgl.][99]{Hamburger1985a}, dann ihr anderes\footcite[\emph{wie hier:} vgl.][33]{Hamburger1980a}, daraus zwei andere Zitate, eins von derselben Seite\footcite[\emph{wie hier:} vgl.][33]{Hamburger1980a}, eins von einer anderen\footcite[\emph{wie hier:} vgl.][105]{Hamburger1980a} und dann wieder eines aus ihrem ersten Buch.\footcite[\emph{wie hier:} vgl.][99]{Hamburger1985a}

Auch die innere Struktur der bibliographischen Angaben sollte lesefreundlich sein: Indem z.B. bei kollektiv erarbeiteten Werken die jeweils letzte Autorin automatisch durch eine Konjunktion eingebunden wird.\footcite[\emph{wie hier}: vgl.][]{SegEvaTay2009a} Oder indem im Literaturverzeichnis -- anders als in der Fußnote selbst -- die erste Autorin per \texttt{Nachname, Vorname} genannt wird, alle folgenden nach dem Muster \texttt{Vorname Nachname}.\footnote{\emph{wie hier:} \cite[bei][]{GarSpr2018a} oder \cite[bei:][]{RusNor2004a}}. Oder indem bei der ersten Erwähnung eines Artikels jeweils der Seitenbereich \emph{und} die je zitierte Seiten angegeben werden, und zwar bei Zeitschriftenartikeln\footcite[\emph{wie hier:} vgl.][51]{Siart2008a} und bei Sammlungsartikeln\footcite[\emph{wie hier:} vgl.][269]{Frobenius1998a}.

Doch nicht nur Bücher oder Sammlungen sollten dem Muster von \emph{Vollnachweis} plus mehrere \emph{kondensierte Kurznachweise} plus Neubeginn mit \emph{expliziertem Kurznachweis} etc. folgen, sondern auch Zeitschriften- und Sammlungensartikel - wobei jeder Seitenwechsel die Folge \emph{kondensierter Kurznachweise} unterbrechen und sie auf der nächsten Seite mit einem erneuten \emph{expliziertem Kurznachweis} wieder neu ansetzen lassen möge:
\begin{description}
  \item[Zeitschriftartikel]: Vollform\footcite[vgl.][50]{Lewin1992a}, ders-ebda\footcite[vgl.][50]{Lewin1992a}, ders-aaO\footcite[vgl.][52]{Lewin1992a}, Zwischenwerk\footcite[vgl.][9]{KantKdU1974}, Kurzform\footcite[vgl.][60]{Lewin1992a}
  \item[Sammlung]: Vollform\footcite[vgl.][X]{CalUrb2013a}, dies-ebda\footcite[vgl.][X]{CalUrb2013a}, dies-aaO\footcite[vgl.][XI]{CalUrb2013a}, Zwischenwerk\footcite[vgl.][9]{KantKdU1974}, Kurzform\footcite[vgl.][X]{CalUrb2013a}
  \item[Beitrag aus einer schon zuvor zitierten Sammlung]: Vollform\footcite[vgl.][83]{Calella2013a}, dies-ebda\footcite[vgl.][83]{Calella2013a}, dies-aaO\footcite[vgl.][94]{Calella2013a}, Zwischenwerk\footcite[vgl.][9]{KantKdU1974}, Kurzform\footcite[vgl.][X]{Calella2013a}
  \item[Beitrag aus bisher unzitierter Sammlung]: Vollform\footcite[vgl.][24]{Flender1998a}, dies-ebda\footcite[vgl.][24]{Flender1998a}, dies-aaO\footcite[vgl.][25]{Flender1998a}, Zwischenwerk\footcite[vgl.][9]{KantKdU1974}, Kurzform\footcite[vgl.][24]{Flender1998a}

  \item[Proceeding]: Vollform\footcite[vgl.][X]{Brachman1985a}, dies-ebda\footcite[vgl.][X]{Brachman1985a}, dies-aaO\footcite[vgl.][XI]{Brachman1985a}, Zwischenwerk\footcite[vgl.][9]{KantKdU1974}, Kurzform\footcite[vgl.][X]{Brachman1985a}
  \item[Artikel aus schon zuvor zitierten Proceedings]: Vollform\footcite[vgl.][83]{Hays1985a}, ders-ebda\footcite[vgl.][83]{Hays1985a}, ders-aaO\footcite[vgl.][94]{Hays1985a}, Zwischenwerk\footcite[vgl.][9]{KantKdU1974}, Kurzform\footcite[vgl.][X]{Hays1985a}
  \item[Artikel aus bisher unzitierter Proceedings]: Vollform\footcite[vgl.][45]{Woods1991a}, ders-ebda\footcite[vgl.][45]{Woods1991a}, ders-aaO\footcite[vgl.][46]{Woods1991a}, Zwischenwerk\footcite[vgl.][9]{KantKdU1974}, Kurzform\footcite[vgl.][46]{Woods1991a}
\end{description}


\tit{Offensichtlich} lassen sich meine Wünsche mittlerweile komplett in und mit \LaTeX, Bib\LaTeX\ und \emph{biblatex-dw} umsetzen.\footnote{jedenfalls fast. Genau betrachtet, zeigt sich noch dies: \\
1. hat Jabref (5.x) momentan noch den Bug, dass es den Inhalt des crossref-Feldes des Frontends nicht in die bib-Datei übernimmt. Hier müssen wir diese Referenz manuell über den Reiter \emph{biblatex source} eintragen.\\
2. unterscheidet das System noch nicht zwischen neuen Artikeln aus neuer/n Collection / Proceedings und neuen Artikeln aus schon vorab zitierter/n Collection / Proceedings: bei einem neuen Artikel werden auch alle Daten der Sammlung mit ausgegeben, selbst wenn die Sammlung als solche schon mehrfach verwendet worden ist. Aber damit kann ich leben.\\
3. ist der Umgang mit den Kürzeln \emph{f}, \emph{ff} oder \emph{et passim} noch nicht ganz perfekt geregelt: Findet sich im Feld \emph{pages} der Bibliographie oder in der \emph{postnote} eines Zitates nur eine Zahl oder ein Zahlbereich wie \texttt{23-26}, dann fügt Bib\LaTeX\ automatisch das Kürzel \emph{S.} bzw. im Englischen \emph{p.} und  \emph{pp.} ein. Schreibt frau in die \emph{postnote} jedoch  \texttt{23f}, \texttt{23ff} oder gar \texttt{23 et passim}, unterdrückt  Bib\LaTeX\ das vorgestellte Seitenkürzel. Hier ist mir noch nichts besseres begegnet, als in diesen Fällen das Seitenkürzel manuell mit in die \emph{postnote} eines Zitates aufzunehmen (\emph{wie hier}: \cite[vgl.][S. 50f et passim]{Lewin1992a})
}

Es gibt also keinen Grund mehr, selbst mit \LaTeX-Stildatein herumzuspielen. Und bei sauberer Erfassung der Daten ist auch das \emph{gender}-Problem längst keines mehr. Ich kann gar nicht genug betonen, wie dankbar ich für Bib\LaTeX\ und \tit{biblatex-dw}, wie dankbarbar ich  \emph{Dominik Waßenhoven} bin.\footnote{Neben dem Handbuch für {biblatex-dw} (\cite[vgl.][]{Waßenhoven2016a}) hat \emph{Dominik Waßenhoven} in zwei Artikeln die Artbeit mit Bib\LaTeX\ erläutert (\cite[vgl.][]{Waßenhoven2008a} u. \cite[vgl.][]{Waßenhoven2008b}). Allerdings geht es ihm (auch) darum, den Umgang mit der Variantenvielfalt von Bib\LaTeX\ und \tit{biblatex-dw} darzulegen, mir dagegen nur darum, einen Weg zu einem ausgefeilten geisteswissenschaftlichen Anmkerungsapparat zu demonstrieren.}

Offen ist aber noch, warum ich so gestaltete wissenschaftliche Texte vorziehe und wie frau sie mit wenig Handarbeit erzeugen kann.



%\bibliography{../bib/literature}

% This file is part proScientia.ltx/humanities.
% (c) 2022 K. Reincke (https://github.com/kreincke/proScientia.ltx/humanities)
% Unlike the other files of proScientia, all files under proScientia/humanities
% are licensed under the terms of the creative commons license CC-BY-SA-4.0
% (= https://creativecommons.org/licenses/by-sa/4.0/)

\section{Was der Zweck ist.}

Warum Wissenschaftlerinnen jede Übernahme als Zitat ausweisen sollen, begründet das deutschsprachige \enquote{Standardwerk} zum Thema \enquote{wissenschaftlichen Arbeiten}\footnote{\cite[vgl.][S. 159ff]{Theisen2013a}. Dass dieses Buch 2013 in der 16. Aufl. erschienen ist und mittlerweile in der 18. Aufl. bei angeboten wird (s. \lnkb{https://www.amazon.de/dp/3800663732/}{2022-02-13}, unterstreicht, das es wirklich ist, was sein Schmutzumschlag zu sein behauptet, nämlich ein \enquote{Standardwerk}.} legalistisch und funktional: einerseits verlange das Urhebergesetz solche Belege\footcite[vgl.][159]{Theisen2013a}, andererseits müsse \enquote{sichergestellt} sein, \enquote{[...] dass für wissenschaftliche Zwecke nur solches Material verwendet wird, das nachvollziehbar und damit auch \textbf{kontrollierbar} ist}.\footcite[vgl.][160]{Theisen2013a} Danach beschreibt dieses Buch verschiedene Nachweisformen\footnote{so den \enquote{Vollbeleg} (\cite[vgl.][S. 161f]{Theisen2013a}), den \enquote{Kurzbeleg}(\cite[vgl.][S. 163f]{Theisen2013a})}, diskutiert die Positionen der Nachweise im zitierenden Text\footcite[vgl.][S. 166ff]{Theisen2013a} und klassifiziert Zitate von ihrem Verhältnis zum Original her\footnote{so das \enquote{direkte} bzw. \enquote{wörtliche} Zitat (\cite[vgl.][S. 169ff]{Theisen2013a}), das \enquote{indirekte} Zitat (\cite[vgl.][S. 174ff]{Theisen2013a}), das \enquote{Sekundärzitat}  (\cite[vgl.][S. 177f]{Theisen2013a}) und das \enquote{Zitat im Zitat}(\cite[vgl.][179]{Theisen2013a})}. Und für all diese Aspekte hatte es vorab konstatiert, dass \enquote{die genaue Kenntnis und sorgfältige Berücksichtigung der unterschiedlichen \textbf{Zitatformen} [...] eine 'conditio sine qa non' (lat.: ein zwingendes Erfordernis) (sei) [...]}\footcite[vgl.][S. 159 {herv. u. übers. i.O.}]{Theisen2013a}.

Leider ist die Aussage, die 'Berücksichtung solcher Kenntnisse' sei eine 'condition sine qua non' vorderhand nur eine Behauptung. Als solche kann sie Zweck und Form des richitgen Zitierens nicht begründen. Hilfreicher ist da schon der Hinweis auf die 'Kontrollierbarkeit'. Ich sehe vier Gründe, Aussagen anderer zu zitieren\footnote{vgl. dazu \cite[][52]{ModLanAss2009a} und \cite[][187]{RueStaFra1980a}. Letztere beschreiben die Funktionen ähnlich, legen aber andere Schwer\-punk\-te: So läuft das, was ich als affirmatives Zitat bezeichnen, bei ihnen als 'Bestätigung wissenschaftlicher Thesen durch anerkannte Autoritäten oder Arbeiten', während das, was ich als 'konfrontatives Zitat' bezeichne, bei Ihnen nicht vorkommt.}:

\begin{enumerate}
  \item Jemand anderes hat einen Befund oder Ansatz geliefert, dessen Wahrheit, Gültigkeit oder Relevanz ich in Zukunft voraussetze. Ich referiere diesen Befund über \emph{affirmative Zitate}, baue meine Argumentation darauf auf und kann so einen Teil meiner Arbeit 'delegieren'. Mithin ist es in meinem Interesse, diese Vorarbeit 'kontrollierbar' zu zitieren, damit ich selbst in meiner Argumentation den Schritt weg von der bloßen Behauptung hin zum verifizierbaren Argument tue.
  \item Jemand anderes hat einen Befund oder Ansatz geliefert, dessen Wahrheit, Gültigkeit oder Relevanz ich bestreite. Über \emph{konfrontative Zitate} referiere ich zunächst diesen Befund, um ihn anschließend zu widerlegen. Wiederum ist es also in meinem Interesse, diese Vorarbeit 'kontrollierbar' zu zitieren, schärft doch die belegte Abgrenzung meinen eigenen Ansatz.
  \item Jemand anderes hat einen Begriff oder ein Wort benutzt, das ich
  übernehmen will. Damit delegiere ich die Arbeit der Definition an diesen
  anderen und referiere seine Ergebnisse über \emph{adaptive Zitate}. Erneut dient das überprüfbare Zitieren meiner eigenen Arbeit, kann ich bei Rückfragen auf den eigentlichen Schöpfer verweisen.
  \item Ich verweise - grosso modi - auf konkurrierende Positionen, andere
  Aspekte oder erweiterte Kontexte, um sie über
  \emph{abweisende Zitate} aus meinem Fokus auszugrenzen. Das genaue Zitieren macht hier überprüfbar, ob diese ausgegrenzten Aspekte 'abseitig' sind. Denn genau das habe ich ja dadurch behauptet, dass ich nur grosso modi auf sie verwiesen habe.
\end{enumerate}

Man sieht\footcite[vgl. dazu auch][hier werden im Abschnitt
'Wissenschaft' drei Funktionen aufgelistet. Das, was ich als 'affirmatives
Zitat' bezeichne, läuft - unter dem Schlagwort 'auf den Schultern von Riesen' -
als Redundanzreduktion, gepaart mit der Überprüfbarkeit. Außerden wird die Moral
ins Feld geführt]{Wikipedia2011a}: es ist bei jeder dieser Zitatfunktionen in
meinem Interesse, meine Quellen nicht nur 'irgendwie' anzugeben,
sondern es meinen Leserinnen leicht zu machen, sie wiederzufinden. Erschwere ich
es ihnen hingegen, schwäche ich meine Argumentation, schwäche ich mich. Denn
dann könnten sie bestenfalls über mich sagen: 'nun gut, er hat es zumindest
behauptet, aber ob's stimmt, wer weiß? - wir konnten es jedenfalls nicht
wirklich nachprüfen'.

Den argumentativen Funktionen des Zitats stehen seine inhaltlichen Formen gegenüber. Zu unterscheiden wäre zwischen \glqq{}wörtlichem\grqq{} oder \glqq{}nicht-wörtlichem\grqq{} Zitat, will sagen: der wortgetreuen und der \glqq{}[\ldots] sinngemäßen Übernahme oder Wiedergabe schriftlicher oder mündlicher Äußerungen anderer\grqq{}.\footcite[vgl.][S. 187f - ohne Frage, dieses ist ein sinngemäßes und kein wörtliches Zitat. Und es ist affirmativ]{RueStaFra1980a} Anderenorts werden sie als \emph{direktes} und \emph{indirektes Zitat} bezeichnet.\footcite[vgl.][]{WisArbOrgZitate} Persönlich würde ich hier noch feiner unterscheiden und diesen beiden Formen das \emph{begriffliche Zitat} zur Seite stellen:

\begin{description}
  \item[direktes Zitat] :- die wort- und zeichengetreue Wiedergabe (mindestens) eines Satzes (Aussage), ggfls. durch markierte Auslassungen 'konzentriert'. Der Zitator erhebt den Anspruch, exakt wiedergegeben zu haben.\footcite[vgl.][187f]{RueStaFra1980a} Auf die Quelle wird am Ende des in Anführungszeichen eingeschlossenen Textes direkt verwiesen.\footcite[vgl.][172]{Theisen2013a}
  \item[indirektes Zitat] :- eine Paraphrase, die (mindestens) einen Satz (Aussage) sinngemäß wiedergibt. Sie darf einzelne Termini oder Satzteile aus dem Original entnehmen, sofern sie diese mit Anführungszeichen markiert. Auf die Quelle wird am Ende der Paraphrase mit einer Fußnote verwiesen, in der die bibliographischen Daten mit \emph{vgl.} oder \emph{s.} eröffnet werden. Damit wird das direkte vom indirekten Zitat unterschieden.\footcite[vgl.][174]{Theisen2013a} Es signalisiert den Anspruch des Zitators, die Aussage als Ganzes sinngemäß, aber nicht wörtlich wiedergegeben zu haben.\footcite[vgl.][letzter Absatz aus Abschnitt 'Grenzen der Zitierpflicht']{Wikipedia2011a}
  \item[begriffliche Zitat] :- die wort- und zeichengetreue Übernahme eines Wortes bzw. einer Satzkonstituente als ein Begriff. Dieser übernommene Begriff wird in Anführungszeichen gesetzt, auf die Quelle wird unmittelbar nach dem Wort mit Hilfe von {\itshape vgl.} verwiesen. Der Zitator beansprucht damit, die Definition von jemand anderem übernommen zu haben, die Aussage, in die das Übernommene eingebettet ist, aber selbst zu verantworten.
\end{description}

Damit können wir die einfache Frage stellen, welche Zitatformen für welche
Zitatfunktionen dienlich sind:

\begin{center}
\begin{tabular}{|r||c|c|c|}
\hline
& {Direktes Zitat}
& {Indirektes Zitat}
& {Begriffliches Zitat}
\\
\hline \hline
\emph{affirmative Zitate}& \checkmark &  \checkmark & \\
\hline
\emph{konfrontative Zitate}&  \checkmark &  \checkmark & \\
\hline
\emph{adaptive Zitate}&  & \checkmark & \checkmark\\
\hline
\emph{abweisende Zitate}&  & \checkmark & \\
\hline
\end{tabular}
\end{center}

Bliebe noch zu klären, zu welchem Zweck solch ausgeklügelte Regeln befolgt werden sollen. Diese Frage wird besonders klar von dem anderen Standardwerk der Wissenschaftsgemeinde beantwortet, nämlich vom 'MLA Handbook for Writers of Research Papers', das sich selbst als 'The Authorative Guide' bezeichnet.\footcite[vgl.][Buchcover]{ModLanAss2009a} Dazu erläutert es zunächst, was Plagiate sind und was sie für die Forschung bedeuten.\footcite[vgl.][S. 52ff]{ModLanAss2009a} Dann erklärt es, wie Zitattexte korrekt erstellt werden\footcite[vgl.][92ff]{ModLanAss2009a}, um anschließend die Form des zugehörigen Beleges\footcite[vgl.][S. 126ff]{ModLanAss2009a} und die dafür konstitutive \glqq{}List of Works Cited\grqq{}, die Literaturliste zu beschreiben.\footcite[vgl.][126ff]{ModLanAss2009a} Und dabei formuliert es einen beeindruckenden Anspruch:

\begin{quote}\glqq{}They [the responsible writers; KR] specify when they refer to another author's ideas, facts, and words, whether they want to agree with, object to, or analyze the source. This kind of documentation not only recognizes the work writers do; it also tends to discourage the circulation of error, by inviting readers to determine for themselves wether a reference to another text presents a reasonable account of what the text says.\grqq{}\footcite[][52]{ModLanAss2009a}
\end{quote}

Zentral ist hier, dass Leserinnen dazu \textit{eingeladen} (und nicht: daran gehindert) werden sollen, Aussagen anderer Autorinnen, die im gerade gelesenen Text zitiert worden sind, \textit{eigenhändig zu überprüfen}, und zwar nicht nur, ob die Aussagen korrekt wiedergegeben sind (das ist 'nur' eine notwendige Voraussetzung), sondern ob sie in die Argumentation auch \textit{valide eingebunden worden sind} und diese stützen. Zu dieser Forderung an Autorinnen sagt das Handbuch schlicht:

\begin{quote} \glqq{}Plagiarists undermine these important public value. Once detected, plagiarism in a work provokes skepticism and even outrage among readers, whose trust in the author has been broken.\grqq{}\footcite[][52f]{ModLanAss2009a} \end{quote}

Ein solcher Schaden - so das Handbuch - entstehe sogar durch 'unbeabsichtigte Plagiate'.\footcite[vgl.][55 - im Original \glqq{}unintenional plagiarism\grqq{}]{ModLanAss2009a} Und diese können leichter 'entstehen', als eine unbedarfte Autorin anzunehmen geneigt ist. Denn so gelte z.B.:

\begin{quote}\glqq{}Presenting an author's wording without marking it as quotation is plagiarism, even if you cite the source.\grqq{}\footcite[][55 (herv.KR.)]{ModLanAss2009a} \end{quote}

Der tiefere Grund für diese 'Dippelschisserei' liegt darin, dass es zum Wesen der Wissenschaft gehöre, an Vorarbeiten anzuknüpfen: Der Zweck eines Forschungspapieres \glqq{}[\ldots] is to synthesize previous research and scholarship with your ideas on the subject\grqq{}. Und wenn das 'Borgen' intentional schon dazugehöre, dann dürfe \glqq{}[\ldots] the material you borrow [\ldots] not be presented as if it were your own creation\grqq{}.\footcite[vgl.][55]{ModLanAss2009a} Klar, dass das unmarkierte Zitat diese Regel verletzt. Denn wie sollte aus der bloßen Quellenangabe geschlossen werden können, welche Wörter übernommen und welche eigene Zutat sind? Eigentlich also kaum noch erwähnenswert, weil implizit unabdingbar, ist dann noch die folgende ergänzende Regel

\begin{quote} \glqq{}[\ldots] you must document everything that you borrow - not only direct quotations and paraphrases but also information and ideas.\grqq{}\footcite[][52f]{ModLanAss2009a} \end{quote}

Vom Zweck her decken sich also meine Wünsche und die Ansprüche des MLA Handbuches. Wir unterscheiden uns, wenn es um die Form geht. Dazu trägt insbesondere eine zentrale Anweisung des MLA Handbuches bei:

\begin{quote} \glqq{}[\ldots] A citation in MLA style contains only enough information to enable the readers to find the source in the works-cited list.\grqq{}\footcite[][127]{ModLanAss2009a} \end{quote}

Das hat radikale Konsequenzen: Werde in einem 'Erzähltext' beispielsweise der Name einer Autorin erwähnt, von der nur ein Werk zitiert wird, dann reiche es, im Erzähltext nach dem Zitat die bloße Seitenzahl anzugeben. Erst wenn es mehrere Werke seien, müsse zusätzlich zur Seitenzahl ein so gekürzter Titel im laufenden Text eingefügt werden, dass das Werk in der Literaturliste wiedergefunden werden könne.\footnote{\cite[vgl.][127]{ModLanAss2009a}. Fairerweise erwähnt das Handbuch aber, der \glqq{}[\ldots] MLA is not the only way to document sources\grqq{} (\cite[vgl.][127]{ModLanAss2009a}) Eine Alternative sei (etwa) der 'APA style', bei dem im laufenden Text Autorin, Jahr und Seitenzahl angegeben werden und als Muster in das Literaturverzeichnis
verweisen. (\cite[vgl.][S. 127f]{ModLanAss2009a})}

Man darf mithin sagen, der MLA-Stil ist konsequent minimalistisch, also schreiberinnenfreundlich, nicht leserinnenfreundlich. Das erschwert es mir, ihn zu verwenden:

\begin{itemize}
  \item Zum ersten wird der Lesefluss, das 'gleichmäßige' Gleiten des Blickes
  über die Zeilen durch oft eben doch längliche Zitatbelege unterbrochen.
  \item Zum zweiten muss ich mir die Informationen aus dem Kontext
  'zusammenklauben', wenn ich ein Zitat überprüfen will. Wo stand noch gleich
  der Autorinnenname? Welche Seitenangabe bezog sich jetzt grad noch auf sein Werk?
  Mich lädt diese Art nicht ein, das Vorgetragene zu überprüfen.
  \item Und zum dritten und entscheidenden: In diesem Stil können mir Autorinnen
  nicht nebenbei die Forschungssgeschichte vermitteln. Dem steht der Minimalismus entgegen, der die forschungsgeschichtlichen Zusatzhinweise und Markanten wie Verlag, Auflage oder Jahr, wie Name der Zeitschrift oder der Serie etc. etc. in den Anmerkungen einfach beiseite lassen muss.
\end{itemize}

Der inhaltlich sicher gute Artikel 'Intellectualism as Cognitive Science' weist genau dieses unruhige Lese-Bild auf, das zwar wissenschaftlich korrekt ist, aber rezeptiv stolpern lässt: Die Zitate werden im laufenden Text innerhalb von Klammern nach dem Schema 'Autorin, Jahr, Seite' belegt.\footcite[vgl.][25]{RotCum2011a} Dabei unterbrechen die Klammern den Lesefluss. Und die Leserinnen, die den Titel des zitierten Werkes und damit die engste Zusammenfassung des Inhalts kennenlernen wollen, müssen ins Literaturverzeichnis blättern.\footnote{\cite[vgl.][S. 38f]{RotCum2011a}. Andere Werke referieren sogar nur noch über 'Autorin und Jahr' und verzichten ganz auf Seitenzahlen. Damit wäre die Überprüfbarkeit nicht nur stilistisch erschwert, sondern gänzlich verloren gegangen (\cite[vgl. z.B.][151]{Bechtel2011a})}

Der Vorteil des MLA-Stiles ist zugleich sein Nachteil: Er behandelt seine Lese\-rin\-nen auf Augenhöhe. Er geht unter der Hand davon aus, dass jede Leserin die (Mehrheit der) zitierten Werke im Prinzip schon kennt. Minimalistisch diskutiert die kundige Autorin mit der lesenden Expertin. Das Problem ist nur: Nicht alle Leserinnen sind immer schon Expertinnen. Und warum sollte eine Wissenschaftlerin es ihrer Leserin nicht erleichtern, selbst zur Expertin zu werden?

Deshalb ziehe ich den (alt)philologisch/philosophischen Zitierstil vor. Er hat mir die Rezpetion erleichtert, nicht nur die eines Werkes, sondern die des Faches selbst. Und in diesem Sinne will ich auch meinen Leserinnen dienen.


%\bibliography{../bib/literature}

% This file is part proscientia.ltx/humanities.
% (c) 2022 K. Reincke (https://github.com/kreincke/proscientia.ltx/humanities)
% Unlike the other files of proscientia, all files under proscientia/humanities
% are licensed under the terms of the creative commons license CC-BY-SA-4.0
% (= https://creativecommons.org/licenses/by-sa/4.0/)

\section{Wie es umgesetzt wird.}

Mein Wunsch über all die Jahre war, dass der altphilologisch geisteswissenschaftliche Schreib- und Argumentierstil \LaTeX-like ermöglicht wird: Ein simpler Befehl für den Zitatbeleg, und der Rest sollte sich von allein
ergeben, gerne mittels Zuladung von Paketen gesteuert, über Konfigurationen
verfeinert und mit Bib\TeX\footcite[vgl.][]{BibtexOrgDe} oder Bib\LaTeX\footcite[vgl.][]{BibLaTeX2022a} umgesetzt. Das Berücksichtigung der fitzeligen Kleinigkeiten eines korrekten bibliographischen Nachweises jedoch wollte automatisiert sehen.

Mit einem modifizierten \emph{Jurabib}\footcite[vgl.][]{Berger2004a} und \emph{KOMA-Script}\footcite[vgl.][]{Kohm2008a} war ich dem schon Anfang 2000 nahe gekommen. Aber eben nicht perfekt. Trotzdem hatte ich die Modifikationen unter dem Namen \emph{mycsrf} in einem Repository samt IDE zur Verfügung gestellt\footcite[vgl.][]{Reincke2021a} und sogar ein passendes Handbuch dazu offeriert.\footcite[vgl.][]{Reincke2018a}

Nun aber ist all dies nicht mehr notwendig.\footnote{eigentlich schon seit spätestens 2016, nur habe ich die bessere Lösung leider erst 2018 entdeckt} Es geht viel einfacher und schlanker, mit \LaTeX, Bib\LaTeX\footcite[vgl.][]{BibLaTeX2022a},  Biber\footcite[vgl.][]{Biber2022a} und \emph{biblatex-dw}\footcite[vgl.][]{Waßenhoven2016a}. Möchte eine Autorin verstehen, wie das zusammenwirkt, muss sie in erster Linie \LaTeX\ kennen und können. Dann kann sie dazu die Quellen dieses Artikels untersuchen oder - noch einfacher - die Templatelösung \emph{proscientia.ltx} auschecken. In aller Kürze skizziert, funktioniert es so:

Zuerst wird im Header des eigenen \LaTeX-Textes das \emph{biber}-Package und der \emph{authortitle-dw}-Stil eingebunden.\footnote{Dieser ist das, was wir meinen, wenn wir von \emph{biblatex-dw} sprechen.} Dann muss eine dazu passende Konfiguration eingebunden werden, die die Details des Erscheinungsbildes festelegt:

\small
\begin{verbatim}
\usepackage[
  backend=biber,
  style=authortitle-dw,
  sortlocale=auto,
]{biblatex}
\ProvidesFile{biblatex.cfg}
% This file is part of proscientia.ltx
% (c) 2022 Karsten Reincke (https://github.com/kreincke/proscientia.ltx)
% It is distributed under the terms of the creative commons license
% CC-BY-4.0 (= https://creativecommons.org/licenses/by/4.0/)


\ExecuteBibliographyOptions{
  %%%%%%%%%%%%%%%%%%%%%%%%%%%%%%%%%%%%%%%%%%%%
  % configurations offered by biblatex itself
  % ------------------------------------------
  maxnames=5,  % Truncate author list after 5 authors ...
  minnames=3,  % ... But display at least 3 authors
  autocite=inline,
  hyperref=true,  % Use hyperref package (should be automatically detected, though)
  backref=true,  % Back references from bibliography page to each encounter
  backrefstyle=two,  % Combine back refs if on two consecutive pages
  isbn=true,  % (Dont) print ISBN, ISSN numbers
  autolang=hyphen,
  % track 'reused' csquotes
  citetracker=constrict, %
  loccittracker=constrict, % discriminate different pages (a.a.O) versues same page (ebda)
  opcittracker=constrict,
  idemtracker=constrict,
  ibidtracker=constrict,
  pagetracker=true,
  %%%%%%%%%%%%%%%%%%%%%%%%%%%%%%%%%%%%%%%%%%%%
  % configurations offered by authortitle-dw
  % ------------------------------------------
  annotation =true,
  namefont=normal,
  firstnamefont=normal,
  idemfont=italic,
  ibidemfont=italic,
  idembib=true, % cluster the books of the same author in bib
  idembibformat=idem, % indicate the same author by ders/dies.
  editorstring=brackets, % parens=(Hrsg.) | brackets=[Hrsg.] | normal = , Hrsg.
  nopublisher=false, % insert publisher into bib data
  editionstring=true, % allow strings in the edition field
  % ZNAME VOL (YEAR) Nr. YOURNALNUMBER
  journalnumber=afteryear, %
  %journumstring=h.
  series=afteryear,
  seriesformat=parens,
  addyear=true, %insert year after titel in 'shorttitle' hints => no year in shorttitle
  firstfull=true, % frist quote complete
  edstringincitations=false, %editor and translator only in the first
  citepages=separate, % vollzitat erst mit seiten, dann 'hier: S. 12'
}
% Put your definitions here.

% refine seriesformat by adding a prefix inside of the parens
\renewcommand*{\seriespunct}{=\addspace}

% by default Biblatex-dw uses Autor1/Autor2 in cites
% these redefinitions overwrite that behaviour by
% duplicating the style used for the bibliography
% (S. p. 35 in the German biblatex-dw handbook )
\renewcommand*{\citemultinamedelim}{\addcomma\space}
\renewcommand*{\citefinalnamedelim}{%
\ifnum\value{liststop}>2 \finalandcomma\fi
\addspace\bibstring{and}\space}%
\renewcommand*{\citerevsdnamedelim}{\addspace}

% biblatex-dw does not print 'ders.' / 'dies.' in a row of cites quoting
% the same book. Additionally, it does not know the diffrence between the
% same page of of the same work and a different page of the same work
% quoted before.
%
% As soon as biblatex 3.15 is offered by UBUNTU
% use \bibncpstring[\mkibid]{ibidem} instaed of inserting the
% German string literally as this hack does:
\renewbibmacro*{cite:ibid}{%
{\ifthenelse{\ifloccit}
  {\printtext[bibhyperref]{\usebibmacro{cite:idem} \mkibid{ebda}}%
   \global\booltrue{cbx:loccit}}
  {\printtext[bibhyperref]{\usebibmacro{cite:idem} \mkibid{a.a.O, }}}
}
}%
\DefineBibliographyStrings{english}{%
  urlseen = {reviewed},
}
\DefineBibliographyStrings{german}{%
  urlseen = {heruntergeladen am},
}
\endinput

\end{verbatim}
\small

\emph{proscientia.ltx} stellt eine solche Konfigurationsdatei so bereit, wie sie zur Erzeugung dieses Textes verwendet worden ist. Jedenfalls harmoniert so auch diese neue Lösung von \emph{Dirk Waßenhoven} wunderbar mit
\emph{KOMA-Script}.\footcite[vgl.][]{Kohm2008a}

Schließlich muss im Header mittels des Befehls \texttt{\textbackslash{}addbibresource\{xyz.bib\}} noch festgelegt werden, welche Bibliotheksdateien ausgewertet werden sollen.

Hat die Autorin dann einen wissenschaftlichen Text mit Fuß- oder Endnoten erstellt, kann er leicht als PDF kompiliert werden:

In der Bibliographiedatei wurden die bibliographischen Daten der zitierten Werke ja zu Gruppen zusammengefasst und je Werk einem Identifier versehen. Im zitierenden Text schlägt das \emph{footnote}-Tag die Brücke: \texttt{\textbackslash{}footcite[vgl.][25]\{Kohm2008a\}}. Und zuletzt bringt die Aufrufsequenz \texttt{latex biber latex latex} die Dinge zusammen. \emph{proscientia.ltx} stellt für einen automatisierten Ablauf ein \emph{Makefile} bereit.

\section{Womit es zusammengestellt wird.}

Zu bedenken wäre noch, womit eine Autorin bibliographische Angaben in ihre Bibliographiedatei eintragen soll und welche Felder pro Eintrag dazugehören. Dies hängt vom Typ eines Werkes, der bibliothekarischen bzw. bibliographischen Tradition und vom Bibliographiestil ab - hier also von \emph{biblatex-dw}-Konfiguration. Der berücksichtigt einerseits das angestrebte Erscheinungsbild, andererseits die Nachweistradition. So entstehen drei Fragen:
\begin{itemize}
  \item \emph{Welche Literaturtypen braucht die Geisteswissenschaftlerin?}
  \item \emph{Welche Datenfelder gehören typischerweise zu einem Literaturtyp?}
  \item \emph{Und wie trägt frau diese am besten in die Bibliographie-Datei ein?}
\end{itemize}

Gehen wir das der Reihe nach an:

Bib\LaTeX\ selbst stellt eine große Menge möglicher Typen bereit und definiert die dazugehörenden Datenfelder.\footcite[vgl.][S. 8ff]{BibLaTeX2021a} Die Semantik der Felder erläutert es gesondert.\footcite[vgl.][S. 33ff]{BibLaTeX2021a} Es strebt an, möglichst viele Veröffentlichungsformen aus möglichst vielen Wissenschaften abzudecken. Das ist systemisch sinnvoll, aus Sicht einer spezifischen Wissenschaftlerin jedoch beschwerlich: sie muss zu viel für sie unnützes Wissen aussortieren. Wäre es nicht nutzbringender, bekäme sie vorab eine Liste der in ihrem Fach gängigen Literaturtypen? Wäre es nicht praktischer, den dann im Einzelfall doch noch fehlenden Typ später 'nachzukonfigurieren', als immer wieder die Masse auf das je Benötigte einzudampfen? Wäre es nicht semantisch sauberer und damit leichter zu merken, unterschieden sich diese Literaturtypen über die Pflichtfelder? Und würde so eine Ordnung die Nutzung eines graphischen Frontends wie \emph{JabRef} nicht erleichtern?

Keine Frage: Eine solche Zusammenstellung zu liefern, kann nicht die Aufgabe von Bib\LaTeX\ oder \emph{biblatex-dw} sein, wohl aber die einer auf die (Alt)Philologie/Philosophie ausgerichteten Konfiguration\footnote{die bruchlos auch die Geschichts- und Musikwissenschaft mit abdeckt}, die der Nutzerin abnehmen möchte, was sie sich sonst aus eigenem Antrieb auch erzeugen würde. Ein guter Weg dahin beginnt damit, die allen Literaturtypen gemeinsamen \ttt{fakultativen} bibliographischen Angaben abzusondern. Die dann verbleibenden typespezifischen Mengen von obligatorischen und fakultativen bibliographischen Angaben sollten paarweise disjunkt sein\footnote{Die in der Geisteswissenschaft üblichen sechs Grundtypen sind sogar nur bezogen auf die zwingend erforderlichern Angaben disjunk.}:

\begin{longtable}{|r||p{4,5cm}|p{5cm}|}
\hline
& zwingend & optional
\\
\hline \hline
\ttt{article}\footnote{Zeitschriftenartikel} & author, gender, title, journaltitle, date, pages & journalsubtitle, number, series, volume, issn, issue \\
\hline
\ttt{book}\footnote{Buch einer oder mehrerer Autorinnen, die (als Team) für größere Teile des Buches stehen} & author, gender, title, location, date &  edition, editor, isbn, publisher, volumes, volume, series, number\\
\hline
\ttt{collection}\footnote{Sammlung von Artikeln, wie z.B. eine Festschrift oder Handbuch} & title, editor, gender, location, date & edition, isbn, publisher, volumes, volume, series number \\
\texttt{misc}\footnote{alle Sonderfälle} & & author, gender, title, editor, location, year \\
\hline
\ttt{online}\footnote{Seite aus dem Internet} & url, urldate & author, gender, title, organization, location, year \\
\hline
\ttt{proceedings}\footnote{Konferenzbericht} & title, editor, gender, location, date, organization & edition, isbn, publisher, volumes, volume, series number\\
\hline \hline
\end{longtable}

Dazu müssten noch die folgenden abgeleiteten Typen kommen, die zwar dieselbe Signatur an Feldern haben, darin aber auf unterschiedliche 'Primärtypen' refererien\footnote{An dieser Stelle ein Hinweis: Ich weiß, dass man beim Sammlungsartikel und beim Konferenzbericht die bibliographischen Angaben des jeweiligen Bandes mit denen des jeweiligen Artikels zusammenfassen kann. Mir erscheint es leichter, wenn frau die Bände jeweils als eigene Einträge in der Literaturliste auflisten lässt und also die Artikel per \ttt{crossref} darauf verweisen lässt: das reduziert Fehlerquellen und liest sich einheitlicher.}:

\begin{longtable}{|r||p{4,5cm}|p{5cm}|}
\hline
& zwingend & optional
\\
\hline \hline
\ttt{inbook}\footnote{Abschnitt eines Buches, das von anderen Autorinnen als denen des Buches als solches geschreiben worden ist} & author, gender, title, pages, crossref &  \\
\hline
\ttt{incollection}\footnote{Artikel in einer Sammlung} & author, gender, title, pages, crossref &  \\
\hline
\ttt{inproceedings}\footnote{Beitrag in einem Konferenzbericht} & author, gender, title, pages, crossref &  \\
\hline \hline
\end{longtable}

Zudem sind bei jedem Literaturtyp noch die folgenden generellen optionalen Felder zulässig: \ttt{shorttitle, subtitle, titleaddon, translator, addendum, annotation, note, langid, url, urldate, owner, timestamp, file, keyword, abstract}.

Mit diesen Festlegungen ausgestattet, sollte eine Autorin ihr Literaturverwaltungsprogramm, das ihre  Bib\LaTeX-Bibliographie-Datei schreibt, noch entsprechend konfigurieren. Falls sie auch mit \emph{JabRef}\footcite[vgl.][wp]{Jabref2019a} arbeitet, geschieht das so:

\begin{itemize}
  \item Upgrade von JabRef über das Betriebssystem auf die Version 5.x.
  \item Umstellung des Bibliographiedateityps bei jeder neu angelegten Bibliographiedatei auf das Bib\LaTeX-Format.\footnote{Über \emph{Library/Library properties} kann frau in \emph{JabRef} festlegen, dass ihre Bibliographiedatei den Bib\LaTeX-Stil verwenden soll.}
  \item Erweiterung der Liste der generell für alle Literaturtypen zulässigen Felder.\footnote{Über \emph{prefrences/Custom Editor Tabs} in \emph{JabRef} werden die Felder definiert, die jeder Eintrag mit bringen soll und die darum in einem Feld erscheinen.}
  \item Begrenzung der Felder der erwähnten Literaturtypen auf die gelisteten Signaturen.\footnote{Über \emph{Options/Customize entry types}  in \emph{JabRef} legt sie fest, welche Daten ein typgerechter Datensatz (über die generellen hinaus) mitbringen soll.}
\end{itemize}

Auch hier hilft \emph{proscientia.ltx}: Im Ordner \emph{cfg} finden sich die Dateien \texttt{jabref-prefs.xml} und \texttt{jabref-customizedBiblatexTypes-prefs.xml}, die die entsprechenden Einstellungen mitbringen.

\emph{JabRef} wertet zwei Konfigurationsdateien aus, die unter \texttt{./.java/.userPrefs/org/} oder unter \texttt{./.java/.userPrefs/net/sf/}  liegen\footnote{welche das je eigene System zieht, muss frau ausprobieren: Dazu startet sie \emph{JabRef} einmal und sieht mit dem Befehl \ttt{find ./.java | grep jabref} nach, wo die Hauptordner entstanden sind.}, nämlich \texttt{jabref/prefs.xml} und \texttt{jabref/customizedBiblatexTypes/prefs.xml}.\footnote{In der ersten Datei wird festgelegt, dass \emph{JabRef} die Bib\LaTeX-Typen verwendet und welche generellen Felder es im Frontend anbietet. In der zweiten Datei werden die Literaturtypen und ihre spezifischen Datenfelder Bib\LaTeX-bezogen definiert.} Die Nutzerin braucht also nur die \emph{proscientia.ltx}-Dateien -- jeweils unter dem Namen \ttt{prefs.xml} an die entsprechenden \emph{JabRef}-Stellen zu kopieren.\footnote{\emph{JabRef} muss dabei ausgeschaltet sein.}

So erhält die Autorin in \emph{JabRef} ein aufgeräumtes Frontend ohne verwirrende Duplikate etc.
%\bibliography{../bib/literature}

% This file is part proscientia.ltx/humanities.
% (c) 2022 K. Reincke (https://github.com/kreincke/proscientia.ltx/humanities)
% Unlike the other files of proscientia, all files under proscientia/humanities
% are licensed under the terms of the creative commons license CC-BY-SA-4.0
% (= https://creativecommons.org/licenses/by-sa/4.0/)

\section{Zusammenfassung}

Fassen wir zusammen: Verglichen mit dem numerischen Verweisen oder kryptischen Schlüsselreferenzen innerhalb des Lesetextes, ja selbst verglichen mit stark verkürzendem Autor-Jahr-Schema bieten uns Bib\LaTeX\ und \tit{biblatex-dw} -- entsprechend konfiguriert - eine lese- und lernbegünstigende Alternative: Der Anmerkungsapparat bedient seine immanente Aufgabe, Zitate zu belegen. Und zugleich liefert er eine forschungshistorische Zuarbeit. Er breitet vor dem Leser vertrackte Aspekte der Wissenschaftsgeschichte aus und reicht damit die schmerzliche Detailarbeit seiner Autorin uneigenützig an die Leserin weiter. Wissen ist hier nicht mehr Macht, Gelehrsamkeit nicht mehr Klientel stabilisierendes Herrschaftswissen, sondern schlichter \emph{Dienst an der Kundin}.

\appendix
\section{Die Last volatiler Quellen: Electronic Resources}

\emph{proscientia.ltx} folgt von der Form her einem traditionellen Ideal. Dessen Zweck ist aber sehr modern: Auch im Zuge der Digitalisierung gilt es zu gewährleisten, dass Zitate -- und damit die Bausteine der eigenen Argumentation -- leicht zu kontrollieren sind.\footnote{Das deutsche Standardwerk dazu fordert, dass die Belege \enquote{kontrollierbar} sein müssen (\cite[vgl.][160]{Theisen2013a}), das anglo-amerikanische will die Leserinnen sogar generell zur Überprüfung einladen. (\cite[vgl.][52]{ModLanAss2009a})}

Um den Zugang zu Quellen -- und damit die Überprüfbarkeit -- garantieren zu können, hat sich über die Jahrhunderte ein arbeitsteiliges Modell entwickelt: Jedes deutschsprachige Buch muss bei der deutschen Nationalbibliothek hinterlegt sein, die Aufnahme in die Nationalbibliografie beruht dann - wie es heißt - auf einer \enquote{\enquote{Autopsie} des eingereichten Buches}.\footcite[vgl.][68]{Theisen2013a} Für andere Länder existieren ähnliche Gewährleistungssysteme. Wenigstens über die nationalen Bibliotheken hinweg gibt es zudem eine Kooperation: sie zusammen erheben den Anspruch, die wichtigste auch nicht deutsche Fachliteratur abzudecken und den Forscherinnen -- nötigenfalls im Austauschverfahren -- zur Verfügung zu stellen.

Damit kann die Forschungsgemeinschaft zwei Aspekte voraussetzen, die zusammen eine dauerhafte, oder wenigstens: sehr langfristige Überprüfbarkeit etablieren:

\begin{itemize}
  \item Jedes (gedruckte) Werk kann anhand genauer Angaben in einem mehr oder minder aufwändigen Verfahren über Bibliotheken beschafft werden.
  \item  Kein (gedrucktes) Werk kann sich 'plötzlich' ändern: Genannte Seitenzahlen verweisen auf Seiten, wo steht, was da schon seit Drucklegung stand.
\end{itemize}

Bei \emph{E-Books} oder \emph{E-Papers} gilt das jedoch nicht (mehr)\footnote{Für bestimmte Formate ergibt sich das schon aus ihrer technischen Definition: So erlaubt etwa das ePub-Format (s. \lnka{https://www.w3.org/publishing/epub32/}) dem Interpreter, will sagen: dem E-Reader, das Dokument entsprechend einer personalisierten Schriftgröße zu rendern. Bei konstanter Größe der Sichtfläche ohne horizontale Scrollmöglichkeit führt das notwendig zu einer veränderten Umbruch und also zu einer geänderten Seitenzählung. Der gern genutzte \tit{Kindle} ist ein gutes Beispiel dafür.}:

Universitätsbibliotheken schaffen bestimmte Literatur 'nur' noch als E-Book oder als E-Paper an. Das ist ökonomisch sinnvoll. Es reduziert Lager- und Verwaltungskosten. Allerdings 'kaufen' die Bibliotheken oft nicht die elektronischen Kopien selbst, um diese von ihrem eigenen Server aus an die Leserinnen zu verteilen. Vielmehr erwerben sie von den Verlagen 'nur' das Recht, dass die über die Bibliothek authentifizierten Leserinnen sich eine Kopie aus dem Verlagsnetz downloaden dürfen.\footnote{Das ist im Übrigen ein technischer Grund dafür, warum ein solches Downloaden seitens der Verlage 'nur' aus dem Universtätsbibliotheksnetz heraus ermöglicht wird. Diese Hintergründe sind mir bei verschiedenen Veranstaltung zur Bibliotheksnutzung in Frankfurt und Darmstadt bestätitgt worden. Eine zitierfähige Beschreibung des Verfahrens steht -- laut Auskunft der Bibliothekarinnen (erfragt zuletzt am 26.02.2016) -- jedenfalls für Frankfurt nicht so einfach zur Verfügung.}

Damit wird die 'Verlässlichkeitskette' beschädigt: Verlage können nun einfach und 'unerwähnt' unter demselben Downloadlink die modifizierte Version eines Werkes anbieten. Textstellen könnten damit geändert, Seitenzahlen verschoben worden sein. Das muss nicht einmal absichtlich geschehen. Schon simple technische Fehler können solche Modifikationen zu bewirken.

Verschärft wird das Problem wird dadurch, dass man den zitierten 'E-Werken' nachträgliche Eingriffe nicht ansieht - ganz im Gegensatz zu gedruckten Werken, deren erkennbare Beschädigungen dann auch ihre Belegkraft diskreditieren.

So erschüttert die Nutzung von E-Quellen die Reproduzierbarkeit; sie untergräbt die für die Wissenschaftlichkeit konstitutive Verlässlichkeit der Forschungsliteratur: Dass E-Books und E-Papers 'stillschweigend' ersetzt und ihnen Änderungen nicht unbedingt angesehen werden können, macht es angreifbar, sie zu zitieren. Denn auch 'korrekt' ausgewiesene Zitate sind dann nicht mehr in einem letzten Sinne 'reproduzierbar' und also überprüfbar.\footnote{In diesem Zusammenhang wird erkennbar, dass Bibliotheken als staatliche Einrichtungen mit ihrer Aufgabe, den Bestand und die Beständigkeit der Forschungsliteratur zu horten, auch heute noch eine wissenschaftskonstitutive Funktion wahrnehmen.}

Noch deutlicher tritt dieser systematische Makel dort zu Tage, wo nicht einmal mehr Verlage hinter den über das Internet distribuierten E-Books stehen: Das \enquote{größte Risiko bei der Verwendung von im Internet generierten [\ldots] Materialien} liege - nach Ansicht des deutschen Standardwerks - darin, dass \enquote{[\ldots] nicht jeder Dateneingeber [\ldots] sich bzw. seine Dokumente nachhaltig vor Manipulationen schützen (könne)}\footcite[vgl.][85]{Theisen2013a}: \enquote{die Offenheit des Internet-Systems (erlaube) es, Nachrichten und \emph{Daten} zu verändern oder ganz \emph{zu verfälschen}}. Das bedeute, dass das Internet der \enquote{Flüchtigkeit des Mediums} und der Volatilität der URLs wegen \enquote{[...] nur im Ausnahmefall eine Nachprüfung der Informationen über einen längeren Zeitraum (zulasse)}\footcite[vgl.][S. 86f (herv.i.O)]{Theisen2013a}.

Und dennoch muss dieses Dilemma gelöst werden.\footnote{Glücklicherweise existieren bereits praktische Hinweise für einen guten Umgang mit 'Internetquellen': So enthält etwa das \emph{MLA Handbook for Writers of Research Papers} ein ganzes Kapitel zum \emph{Zitieren von Webpublikationen}. Die Conclusio ist hier, dass frau das besondere Format der Quelle durch ein Kürzel 'Web' in den bibliographischen Daten explizit macht, dass sie die URL des
zitierten Dokumentes hinzufügt und dass sie auch ihr je spezifisches Abrufdatum
in die bibliographischen Angaben integriert (\cite[vgl.][S. 28ff u. 181ff]{ModLanAss2009a}). Das deutsche Standardwerk geht noch einen Schritt weiter. Nachdem es die auf längere Zeit gesehen nur eingeschränkte Nachprüfbarkeit von Internetzitaten hervorgehoben hat, konstatiert es, dass \enquote{[...] elektronische Daten [\ldots] nachhaltig nachgewiesen werden (müssen), so dass der Leser (oder Prüfer) sie auch zu jedem späteren Zeitpunkt \emph{nachvollziehen} kann} (\cite[vgl.][S. 86f]{Theisen2013a}). Als Möglichkeiten für eine solche Verstetigung wird dann auf die \enquote{'Screenshot'-Technik} verwiesen (\cite[vgl.][80 u. 87]{Theisen2013a}) Ich werde so gleich eine verfeinerte Mixtur beider Ansätze als Lösung vorschlagen.}. Denn die E-Werke erfüllen fraglos das Kriterium der Zitierfähgikeit, sofern eben \enquote{[\ldots] alle Quellen und Sekundärmaterialien (zitierfähgig sind), die \emph{in irgendeiner Form} [\ldots] \emph{veröffentlicht} worden sind}.\footcite[vgl.][S. 160 (herv. K.R)]{Theisen2013a} Sie zu ignorieren, ist mithin keine Option.

Um einen potentiellen Mangel an Überprüfbarkeit transparent zu machen und
möglichst auszugleichen, bietet sich folgendes Verfahren an:

\begin{itemize}
  \item Bei gedruckten Werken, die sich eine Autorin über das normale Bibliotheks- resp. Verlagssystem beschafft und \tit{eigenhändig} ausgewertet hat, möge sie die bibliographischen Angaben im Literaturverzeichnis\footnote{unter Ausnutzung des Bib\LaTeX\ Tokens \ttt{note}} mit dem Schlagwort \emph{Print} markieren. Damit soll die Verfasserin nicht nur sagen, das Werk selbst 'physisch' eingesehen zu haben. Sie möge damit zudem markieren, die Angaben zur Quelle so genau spezifiziert zu haben, dass die Beschaffung über das normale Bibliothekssystem im Sinne der erwähnten Arbeitsteilung reproduzierbar ist.
  \item Werke, die eine Autorin über ein Netz in elektronischer Form eingesehen und ausgewertet hat, möge sie im Literaturverzeichnis\footnote{unter Ausnutzung des Bib\LaTeX\ Tokens \ttt{note}}  -- formatgemäß -- mit \emph{[BibWeb$|$FreeWeb]/[PDF$|$HTML$|$\ldots]} markieren. Dabei stehe \emph{BibWeb} für ein durch eine Universitätsbibliothek bereitgestelltes Netz\footnote{das entweder physische Präsenz des Auswertenden in den Bibliotheksgebäuden oder die Nutzung eines entsprechenden VPN voraussetzt}, während \emph{FreeWeb} das frei zugängliche Internet meine. Bei Werken aus dem freien Internet möge eine Verfasserin die URL und das Datum in den Tokens \ttt{url} und \ttt{urldate} vermerken, unter der bzw. an dem sie den elektronischen Text eingesehen hat. Zudem sollte sie in beiden Fällen -- wo irgend möglich -- das eingesehene Werk als elektronische Kopie (PDF) sichern.\footnote{Es ist natürlich klar, dass auch die Autorin, die ihre volatilen Quellen zwecks späterem Nachweis als PDF sichert, diese PDF-Dateien selbst verändern kann - mit einem einfach Text-Editor und etwas PDF-Kenntnis. Mithin wäre ihr Nachweis angreifbar. Allerdings ist schon die bloße Möglichkeit, später die Originalform vorlegen zu können, ein Gewinn, dreht das doch die Beweislast um.} Diese darf sie aus Urheberrechtsgründen natürlich nicht frei weitergeben. Im Streitfall sollte es aber beruhigend sein, genau die Version unter juristisch kontrollierbaren Bedingungen selbst einsehbar machen zu können, die sie ausgewertet hat.
  \item In sehr selten Fällen wird eine Autorin die bibliographischen Angaben nicht verifizieren können, obwohl das Werk in der Forschungsliteratur durchgehend mit diesen Angaben zitiert wird. In solchen Fälle sollte sie es\footnote{unter Ausnutzung des Tokens \texttt{note}} mit \emph{[BibWeb$|$FreeWeb]/REF} markieren. Nach menschlichem Ermessen wird ein solcher Fall aber nur im freien Web entstehen.
\end{itemize}

\emph{proscientia.ltx} unterstützt diese Verfahren von sich aus: Wenn man die o.a. Markierungen unter dem Token \ttt{note} in seine Bibtex-Bibliographie aufnimmt, werden sie an angmessener Stelle in die bibliographischen Angaben integriert. Die bereitgestellte \emph{JabRef}-Konfiguration berücksicht die Aktivierung des Tokens auch.
%\bibliography{../bib/literature}

% This file is part proScientia.ltx/humanities.
% (c) 2022 K. Reincke (https://github.com/kreincke/proScientia.ltx/humanities)
% Unlike the other files of proScientia, all files under proScientia/humanities
% are licensed under the terms of the creative commons license CC-BY-SA-4.0
% (= https://creativecommons.org/licenses/by-sa/4.0/)

\section{Die Lust am Gendern}

Wikipedia rechnet den \enquote{geschlechterbewussten Sprachgebrauch} dem \emph{Gendern} im Allgemeinen zu\footcite[vgl.][]{Wikipedia2022a} und definiert die \enquote{geschlechtergerechte Sprache} an sich als einen Sprachgebrauch, \enquote{der in Bezug auf Personenbezeichnungen [...] die Gleichstellung der Geschlechter in gesprochener und geschriebener Sprache zum Ausdruck bringen will}.\footcite[vgl.][]{Wikipedia2022b}  Bib\LaTeX\ bietet dazu -- anders als noch Bib\TeX -- endlich auch eine automatisierte Version; ich hatte das Eingangs demonstriert\footnote{s.S. \pageref{Gender}}:

Bib\LaTeX\ erlaubt es nämlich, jeden Titel mit dem Tag \ttt{gender} auszustatten. Das Handbuch definiert die Semantik des Feldes als \enquote{das Geschlecht des Autors oder das Geschlecht des Herausgebers, wenn es keinen Autor gibt} und stellt folgende syntaktischen Verfeinerungen über den Wert des Feldes bereit: \enquote{sf (femininer Singular, ein einzelner weiblicher Name), sm (maskuliner Singular, ein einzelner männlicher Name), sn (Neutrum Singular, ein einzelner neutraler Name), pf (femininer Plural, mehrere weibliche Namen), pm (maskuliner Plural, mehrere männliche Namen), pn (Neutrum Plural, mehrere neutrale Namen), pp (Plural, mehrere Namen unterschiedlichen Geschlechts)}.\footcite[vgl.][27]{BibLaTeX2021a} Allerdings benötigt frau dazu noch einen Bibliographiestil, der diese Daten auch auswertet. \emph{biblatex-dw} tut das exezellent. So weit, so klar.

Nun stellt sich jedoch die Frage, mit welchem Geschlechter-Kürzel Internetseiten etc. ausgezeichnet werden sollen, die direkt keinen Autor ausweisen, aber sicher einen oder mehrere haben. Naheliegend wäre es, sie als singuläres Neutrum zu markieren (= \ttt{sn}). Das führt aber zu sprachlich hässlichen Auswüchsen: alle Wikipediaseiten würden im Literaturverzeichnis hintereinandergereiht und mit \emph{dass.} für \emph{dasselbe} eröffnet.

Meine bisher beste Lösung dafür ist die:

\begin{itemize}
  \item Internetseiten ohne jede Autorinnenangabe zeichne ich je als singuläres Femininum (= \ttt{sf}) aus.
  \item Internetseiten, von denen ich weiß, dass sie von einer Gruppe erarbeitet werden, dessen Zusammensetzung ich nicht kenne, markiere ich als (= \ttt{pp})
  \item Internetseiten, die von einer Organisation oder Firma bereitgestellt werden, zeichne ich ebenfalls als singuläres Femininum (= \ttt{sf}) aus.
\end{itemize}

Bei all diesen Varianten verwende ich \ttt{anon.} für \emph{anonymus}\footnote{respektive \emph{anonyma}} als Autorinnenname und hänge dem -- wo möglich -- in eckigen Klammern den Namen der Organisation oder Firma an. Jeder Wikipedia-Eintrag hätte also bei mir also die Autorin \ttt{anon. [Wikipedia]}. So liest sich das implizit generalisierte generische Femininum in deutschen Texten flüssig, in englischen ist es eh egal.

Allerdings entsteht damit eine nächste Herausforderung: Nun werden alle Werke von anonymen Autorinnen genau der einen \emph{anonyma} zugeordnet und im Literaturverzeichnis hintereinander aufgelistet, jeweils eröffnet mit \emph{dies.} Das heißt, dass alle Internetseiten etc. implizit derselben Autorin zugeschrieben werden. Das ist unschön. Es lässt sich aber wenigestens dadurch entschärfen, dass frau im bib-File nach dem Tag \ttt{author} auch noch das Tag \ttt{sortname = \{Wikipedia\}} verwendet. So würden wenigstens alle Wikipedia-Seiten etc. jeweils gesondert geclustert.


\small

% insert the nomenclature here


% This file is part proScientia.ltx/humanities.
% (c) 2022 K. Reincke (https://github.com/kreincke/proScientia.ltx/humanities)
% Unlike the other files of proScientia, all files under proScientia/humanities
% are licensed under the terms of the creative commons license CC-BY-SA-4.0
% (= https://creativecommons.org/licenses/by-sa/4.0/)

% specific abbreviations
\abbr[utb]{UTB}{Uni-Taschenbuch}
\abbr[stw]{stw}{suhrkamp taschenbuch wissenschaft}

% general abbreviations
\abbr[vgl]{vgl.}{vergleiche}
\abbr[aaO]{a.a.O.}{am angegebenen Ort}
\abbr[ebda]{ebda.}{ebenda}
% \abbr[id]{id.}{idem = latin for 'the same', be it a man, woman or a group\ldots}
% \abbr[ibid]{ibid.}{ibidem = latin for 'at the same place'}
% \abbr[lc]{l.c.}{loco citato = latin for 'in the place cited'}
\abbr[wp]{wp.}{webpage = Webdokument ohne innere Seitennummerierung}

\printendnotes
\printnomenclature
\printbibliography

\end{document}
