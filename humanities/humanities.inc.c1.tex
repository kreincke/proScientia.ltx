% This file is part proscientia.ltx/humanities.
% (c) 2022 K. Reincke (https://github.com/kreincke/proscientia.ltx/humanities)
% Unlike the other files of proscientia, all files under proscientia/humanities
% are licensed under the terms of the creative commons license CC-BY-SA-4.0
% (= https://creativecommons.org/licenses/by-sa/4.0/)


\section{Worum es geht.}

\begin{quote}
  \tit{In diesem Quasitutorial\footnote{Dieser Artikel ist im eigentlichen Sinne keine Anleitung, wie Sie -- liebe Leserin -- die (alt)\-philologisch philosophische Nachweis- und Zitiermethode in und mit \LaTeX\ nutzen. Vielmehr ist es eine selbstreferentielle Demo: Sie können an und in dem Text sehen, was gemeint ist. Sie können in den Quelldateien nachsehen, wie es gemacht ist. (\cite[s. dazu][]{Reincke2022a}). Allerdings fange ich nicht bei Null an: Wie Sie \LaTeX\ nutzen, woher Sie es bekommen, wie Sie es installieren und wie Sie einen \LaTeX-Text schreiben, sollten Sie schon wissen.} nutze ich das \tbf{generische Femininum}: Nach 1000 Jahren im Vordergrund der deutschen Sprache steht es den Männern gut an, Frauen auch dort vorzulassen.\footnote{Es ist klar: Um Männer ihrerseits nicht zu benachteiligen, werden Frauen kaum fordern, generell das \emph{generische Femininum} zu nutzen. Aber wir Männer können das von uns aus anbieten. Am besten wäre es überhaupt, Frauen nutzten konsequent das \emph{generische Maskulinum} und meinten Frauen mit, Männer dagegen nutzten immer das \emph{generische Femininum} und meinten Männer mit.} Darum gilt hier: \tbf{Männer sind mitgemeint}.}
\end{quote}

Die (alt)philologisch/philosophische Nachweis- und Zitiermethode nahm mich von Anfang an gefangen, selbst wenn ich diese Liebe zu Fuß- und Endnoten später nicht immer ausleben konnte.

Trotzdem wünschte ich mir ziemlich bald, sie auch in und mit \LaTeX\ umsetzen zu können - obwohl ich wusste, dass \LaTeX\ eben nicht aus der europäischen Geisteswissenschaft heraus entstanden war, sondern aus der computerisierten Mathematik und der anglo-amerikanischen Schreibtradition, wie sie im \textit{Handbook for Writers of Research Papers}\footcite[vgl.][]{ModLanAss2009a} spezifiziert ist.

Meine Leidenschaft für die europäische Tradition ließ mich über die Jahre sogar immer wieder einmal schon existierende Stil- und Bibliotheksdateien ausreizen. Ich versuchte \emph{natib}\footcite[vgl.][]{Daly2000a} anzupassen, dann \emph{custom-bib}\footcite[vgl.][]{Daly2007a} und zuletzt \emph{Jurabib}\footcite[vgl.][]{Berger2004a}. Damit näherte ich mich meinem Ideal wenigstens so weit an, dass ich mit  \emph{mycsrf}\footcite[vgl.][]{Reincke2021a} auch eine auf \LaTeX\, \emph{make} und \emph{bash} gründende \emph{IDE} zusammenstellen konnte. Nur perfekt war diese Methode nie. Doch nun bin ich auf \emph{Dominik Waßenhoven} und seine \emph{biblatex-dw}-Lösung\footnote{\cite[vgl.][]{Waßenhoven2016a}. Dieses Paket enthält auch das von demselben Autor geschriebene Handbuch, das aus dem {ctan}-Package-Frontend heraus ausgerufen werden kann. Links in ctan-Pakete werden \emph{ctan} allderings standardmäßig auf Spiegelserver umgeleitet. Es macht also keinen Sinn, einen in Wirklichkeit ja zufällig gewählten Server über das Feld \ttt{url} im bibliographischen Datensatz als \textbf{die} \emph{biblatex-dw-Handbuch-Quelle} 'anzugeben'. Darum verwende ich die bibliographische Angaben für das Paket im Allgemeinen zugleich auch als die für das Handbuch im Besonderen.} gestoßen.

Eigentlich waren meine Wünsche von Anfang an einfach: Ich mag keine Zitate, die innerhalb des Textes durch esoterische Nummernblöcke[42] oder kryptische Bib\TeX-Keys[Daly2000a] belegt werden. Das sind Stolpersteine im Lesefluss. Zudem liebe ich den Subtext, den ein Anmerkungsapparat aufspannt: ich schätze es, wenn mich Autorinnen über die konzentrierte Argumentation ihrer Haupttexte hinaus lustvoll auf Nebenwegen durch die mäandernde Forschungsgeschichte führen. Dieses schwärmenden Hin und Her zwischen Text und Fuß-\footnote{Hier kann ich -- je nach Lust, Laune oder Notwendigkeit -- mit einem Blick nach unten überprüfen oder komplettieren, was mir notwendig oder sinnvoll erscheint. Anmerkungen bringen mir die Forschungsgeschichte nahe, ganz nebenbei - wenn die Nachweise sprechende Belege sind.} oder Endnoten\endnote{Obwohl sie dieselben Optionen wie Fußnoten bieten, zwingen sie mich jedoch zu blättern - was letztlich doch auch stört.} möchte ich genießen.

Doch es bedarf bestimmter Optionen, wenn die Forschungsgeschichte in einem Anmerkungsapparat wirklich leserinnenfreundlich aufgearbeitet werden soll. So wird eine Autorin in einer Fußnote\footnote{wenn sie z.B. nachweisen wollte, dass der \emph{Online-Duden} das Wort \enquote{Dippelschisser} 02/2022 (noch) nicht kennt (\cite[vgl.][]{DudenOnline2022a}), wohl aber der \emph{offene Thesaurus} [\cite[vgl.][]{OpenThesaurus2022a}] oder die \emph{Site der deutschen Synonyme} (\cite[vgl.][]{DeutscheSynonyme2022a})} oder Endnote\endnote{Nur der Form halber dieselbe Anwendung noch einmal: der \emph{Online-Duden} kennt das Wort \enquote{Dippelschisser} 02/2022 (noch) nicht (\cite[vgl.][]{DudenOnline2022a}), wohl aber der \emph{offene Thesaurus} [\cite[vgl.][]{OpenThesaurus2022a}] oder die \emph{Site der deutschen Synonyme} (\cite[vgl.][]{DeutscheSynonyme2022a}).} gelegentlich mehrere Belege unterbringen wollen. Und die Quellennachweise müssen sprechend aufgeschlüsselt sein, bei der ersten Erwähnung bibliographisch vollständig\footnote{\emph{wie hier}: \cite[vgl.][S. 195ff]{RueStaFra1980a}}, samt aller Autorinnen, allen Titelfeinheiten und der Auflage. Als Leserin möchte ich jedenfalls auch auf Übersetzerinnen und Reihen hingewiesen werden, möchte editorische Sonderfälle erkennen und die ISBN oder ISSN erfahren\footcite[\emph{wie hier}: vgl.][]{Covey2006a} - auf dass mir das Wiederfinden der zitierten Werke möglichst leicht gemacht werde.

Erst wenn im Laufe der Argumentation erneut auf dieselbe Quelle zurückgegriffen wird\footcite[\emph{wie hier}: vgl. dazu][194]{RueStaFra1980a}, reicht mir ein verkürzter Beleg\footcite[\emph{wie hier}: vgl.][]{Covey2006a}, der mich zwanglos auf die mnemotechnisch richtige Bahn bringt, ohne mir eine verklausulierte Geheimsprache aufzunötigen: denn \emph{kurz} meint schließlich nicht \emph{esoterisch}\footnote{Es funktioniert also: Den \emph{Rückriem/Stary/Franck} habe ich gerade bibliographisch ebenso ausführlich ausgewiesen, wie den \emph{Covey}. Anschließend habe ich erneut auf beide Quellen verwiesen, nun aber über die Kurzform \emph{Autorin: Titel. (Jahr)}. Die in Bib\LaTeX\ und \emph{biblatex-dw} eingebundenen Features erledigen das automatisch.}.

Wenn allerdings Autorin und Leserin sich im Haupttext über eine längere Strecke auf eine verfeinerte Analyse fremder Gedankengängen einlassen\footcite[\emph{wie hier}: vgl. etwa][32]{Allen2001a} und dabei \footcite[\emph{wie hier}: vgl.][139]{Allen2001a} verschiedene Passagen\footcite[\emph{wie hier}: vgl.][139]{Allen2001a} desselben Werkes begutachten, dann wäre auch der wiederholte Kurznachweis unnötig geschwätzig. Eigentlich reichten ja kürzere Indikatoren: Nehmen wir an, wir zitierten zuerst eine Stelle aus der Kritik der Urteilskraft\footcite[vgl.][9]{KantKdU1974}, dann eine aus der Kritik der reinen Vernunft\footcite[vgl.][45]{KantKdV1974} und dann zwei andere Passagen daraus\footcite[vgl.][69]{KantKdV1974}, nämlich aus der \emph{Transzendentale Ästhetik}\footcite[vgl.][69]{KantKdV1974}. Dann müßte unser Anmerkungsapparat zunächst die beiden ersten Belege komplett auflisten, wobei er schon den zweiten mit einem \emph{ders.} an den ersten anbinden könnte. Und er sollte als nächstes ein \emph{ders., a.a.O. + neue Seite} enthalten, gefolgt von einem
\emph{ders., ebda}. Gingen wir nun zurück auf die \emph{Kritik der Urteilskraft}\footcite[vgl.][9]{KantKdU1974}, dürfte unser Apparat nicht mehr mit dem Kürzel \emph{ders., a.a.O} arbeiten, sondern sollte mit \emph{ders.: (Kurz)Titel} + Jahr einen neuen Bezugspunkt setzen. Und dasselbe sollte auch mit den Büchern einer Autorin möglich sein, allerdings genderkonform: Zuerst das eine Werk\footcite[\emph{wie hier:} vgl.][99]{Hamburger1985a}, dann ihr anderes\footcite[\emph{wie hier:} vgl.][33]{Hamburger1980a}, daraus zwei andere Zitate, eins von derselben Seite\footcite[\emph{wie hier:} vgl.][33]{Hamburger1980a}, eins von einer anderen\footcite[\emph{wie hier:} vgl.][105]{Hamburger1980a} und dann wieder eines aus ihrem ersten Buch.\footcite[\emph{wie hier:} vgl.][99]{Hamburger1985a}

Auch die innere Struktur der bibliographischen Angaben sollte lesefreundlich sein: Indem z.B. bei kollektiv erarbeiteten Werken die jeweils letzte Autorin automatisch durch eine Konjunktion eingebunden wird.\footcite[\emph{wie hier}: vgl.][]{SegEvaTay2009a} Oder indem im Literaturverzeichnis -- anders als in der Fußnote selbst -- die erste Autorin per \texttt{Nachname, Vorname} genannt wird, alle folgenden nach dem Muster \texttt{Vorname Nachname}.\footnote{\emph{wie hier:} \cite[bei][]{GarSpr2018a} oder \cite[bei:][]{RusNor2004a}}. Oder indem bei der ersten Erwähnung eines Artikels jeweils der Seitenbereich \emph{und} die je zitierte Seiten angegeben werden, und zwar bei Zeitschriftenartikeln\footcite[\emph{wie hier:} vgl.][51]{Siart2008a} und bei Sammlungsartikeln\footcite[\emph{wie hier:} vgl.][269]{Frobenius1998a}.

Doch nicht nur Bücher oder Sammlungen sollten dem Muster von \emph{Vollnachweis} plus mehrere \emph{kondensierte Kurznachweise} plus Neubeginn mit \emph{expliziertem Kurznachweis} etc. folgen, sondern auch Zeitschriften- und Sammlungensartikel - wobei jeder Seitenwechsel die Folge \emph{kondensierter Kurznachweise} unterbrechen und sie auf der nächsten Seite mit einem erneuten \emph{expliziertem Kurznachweis} wieder neu ansetzen lassen möge:
\begin{description}
  \item[Zeitschriftartikel]: Vollform\footcite[vgl.][50]{Lewin1992a}, ders-ebda\footcite[vgl.][50]{Lewin1992a}, ders-aaO\footcite[vgl.][52]{Lewin1992a}, Zwischenwerk\footcite[vgl.][9]{KantKdU1974}, Kurzform\footcite[vgl.][60]{Lewin1992a}
  \item[Sammlung]: Vollform\footcite[vgl.][X]{CalUrb2013a}, dies-ebda\footcite[vgl.][X]{CalUrb2013a}, dies-aaO\footcite[vgl.][XI]{CalUrb2013a}, Zwischenwerk\footcite[vgl.][9]{KantKdU1974}, Kurzform\footcite[vgl.][X]{CalUrb2013a}
  \item[Beitrag aus einer schon zuvor zitierten Sammlung]: Vollform\footcite[vgl.][83]{Calella2013a}, dies-ebda\footcite[vgl.][83]{Calella2013a}, dies-aaO\footcite[vgl.][94]{Calella2013a}, Zwischenwerk\footcite[vgl.][9]{KantKdU1974}, Kurzform\footcite[vgl.][X]{Calella2013a}
  \item[Beitrag aus bisher unzitierter Sammlung]: Vollform\footcite[vgl.][24]{Flender1998a}, dies-ebda\footcite[vgl.][24]{Flender1998a}, dies-aaO\footcite[vgl.][25]{Flender1998a}, Zwischenwerk\footcite[vgl.][9]{KantKdU1974}, Kurzform\footcite[vgl.][24]{Flender1998a}

  \item[Proceeding]: Vollform\footcite[vgl.][X]{Brachman1985a}, dies-ebda\footcite[vgl.][X]{Brachman1985a}, dies-aaO\footcite[vgl.][XI]{Brachman1985a}, Zwischenwerk\footcite[vgl.][9]{KantKdU1974}, Kurzform\footcite[vgl.][X]{Brachman1985a}
  \item[Artikel aus schon zuvor zitierten Proceedings]: Vollform\footcite[vgl.][83]{Hays1985a}, ders-ebda\footcite[vgl.][83]{Hays1985a}, ders-aaO\footcite[vgl.][94]{Hays1985a}, Zwischenwerk\footcite[vgl.][9]{KantKdU1974}, Kurzform\footcite[vgl.][X]{Hays1985a}
  \item[Artikel aus bisher unzitierter Proceedings]: Vollform\footcite[vgl.][45]{Woods1991a}, ders-ebda\footcite[vgl.][45]{Woods1991a}, ders-aaO\footcite[vgl.][46]{Woods1991a}, Zwischenwerk\footcite[vgl.][9]{KantKdU1974}, Kurzform\footcite[vgl.][46]{Woods1991a}
\end{description}


\tit{Offensichtlich} lassen sich meine Wünsche mittlerweile komplett in und mit \LaTeX, Bib\LaTeX\ und \emph{biblatex-dw} umsetzen.\footnote{jedenfalls fast. Genau betrachtet, zeigt sich noch dies: \\
1. hat Jabref (5.x) momentan noch den Bug, dass es den Inhalt des crossref-Feldes des Frontends nicht in die bib-Datei übernimmt. Hier müssen wir diese Referenz manuell über den Reiter \emph{biblatex source} eintragen.\\
2. unterscheidet das System noch nicht zwischen neuen Artikeln aus neuer/n Collection / Proceedings und neuen Artikeln aus schon vorab zitierter/n Collection / Proceedings: bei einem neuen Artikel werden auch alle Daten der Sammlung mit ausgegeben, selbst wenn die Sammlung als solche schon mehrfach verwendet worden ist. Aber damit kann ich leben.\\
3. ist der Umgang mit den Kürzeln \emph{f}, \emph{ff} oder \emph{et passim} noch nicht ganz perfekt geregelt: Findet sich im Feld \emph{pages} der Bibliographie oder in der \emph{postnote} eines Zitates nur eine Zahl oder ein Zahlbereich wie \texttt{23-26}, dann fügt Bib\LaTeX\ automatisch das Kürzel \emph{S.} bzw. im Englischen \emph{p.} und  \emph{pp.} ein. Schreibt frau in die \emph{postnote} jedoch  \texttt{23f}, \texttt{23ff} oder gar \texttt{23 et passim}, unterdrückt  Bib\LaTeX\ das vorgestellte Seitenkürzel. Hier ist mir noch nichts besseres begegnet, als in diesen Fällen das Seitenkürzel manuell mit in die \emph{postnote} eines Zitates aufzunehmen (\emph{wie hier}: \cite[vgl.][S. 50f et passim]{Lewin1992a})
}

Es gibt also keinen Grund mehr, selbst mit \LaTeX-Stildatein herumzuspielen. Und bei sauberer Erfassung der Daten ist auch das \emph{gender}-Problem längst keines mehr. Ich kann gar nicht genug betonen, wie dankbar ich für Bib\LaTeX\ und \tit{biblatex-dw}, wie dankbarbar ich  \emph{Dominik Waßenhoven} bin.\footnote{Neben dem Handbuch für {biblatex-dw} (\cite[vgl.][]{Waßenhoven2016a}) hat \emph{Dominik Waßenhoven} in zwei Artikeln die Artbeit mit Bib\LaTeX\ erläutert (\cite[vgl.][]{Waßenhoven2008a} u. \cite[vgl.][]{Waßenhoven2008b}). Allerdings geht es ihm (auch) darum, den Umgang mit der Variantenvielfalt von Bib\LaTeX\ und \tit{biblatex-dw} darzulegen, mir dagegen nur darum, einen Weg zu einem ausgefeilten geisteswissenschaftlichen Anmkerungsapparat zu demonstrieren.}

Offen ist aber noch, warum ich so gestaltete wissenschaftliche Texte vorziehe und wie frau sie mit wenig Handarbeit erzeugen kann.



%\bibliography{../bib/literature}
