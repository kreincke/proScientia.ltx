% This file is part proScientia.ltx/humanities.
% (c) 2022 K. Reincke (https://github.com/kreincke/proScientia.ltx/humanities)
% Unlike the other files of proScientia, all files under proScientia/humanities
% are licensed under the terms of the creative commons license CC-BY-SA-4.0
% (= https://creativecommons.org/licenses/by-sa/4.0/)

\appendix
\section{Die Last volatiler Quellen: Electronic Resources}

\emph{proScientia.ltx} folgt von der Form her einem traditionellen Ideal. Dessen Zweck ist aber sehr modern: Auch im Zuge der Digitalisierung gilt es zu gewährleisten, dass Zitate -- und damit die Bausteine der eigenen Argumentation -- leicht zu kontrollieren sind.\footnote{Das deutsche Standardwerk dazu fordert, dass die Belege \enquote{kontrollierbar} sein müssen (\cite[vgl.][160]{Theisen2013a}), das anglo-amerikanische will die Leserinnen sogar generell zur Überprüfung einladen. (\cite[vgl.][52]{ModLanAss2009a})}

Um den Zugang zu Quellen -- und damit die Überprüfbarkeit -- garantieren zu können, hat sich über die Jahrhunderte ein arbeitsteiliges Modell entwickelt: Jedes deutschsprachige Buch muss bei der deutschen Nationalbibliothek hinterlegt sein, die Aufnahme in die Nationalbibliografie beruht dann - wie es heißt - auf einer \enquote{\enquote{Autopsie} des eingereichten Buches}.\footcite[vgl.][68]{Theisen2013a} Für andere Länder existieren ähnliche Gewährleistungssysteme. Wenigstens über die nationalen Bibliotheken hinweg gibt es zudem eine Kooperation: sie zusammen erheben den Anspruch, die wichtigste auch nicht deutsche Fachliteratur abzudecken und den Forscherinnen -- nötigenfalls im Austauschverfahren -- zur Verfügung zu stellen.

Damit kann die Forschungsgemeinschaft zwei Aspekte voraussetzen, die zusammen eine dauerhafte, oder wenigstens: sehr langfristige Überprüfbarkeit etablieren:

\begin{itemize}
  \item Jedes (gedruckte) Werk kann anhand genauer Angaben in einem mehr oder minder aufwändigen Verfahren über Bibliotheken beschafft werden.
  \item  Kein (gedrucktes) Werk kann sich 'plötzlich' ändern: Genannte Seitenzahlen verweisen auf Seiten, wo steht, was da schon seit Drucklegung stand.
\end{itemize}

Bei \emph{E-Books} oder \emph{E-Papers} gilt das jedoch nicht (mehr)\footnote{Für bestimmte Formate ergibt sich das schon aus ihrer technischen Definition: So erlaubt etwa das ePub-Format (s. \lnka{https://www.w3.org/publishing/epub32/}) dem Interpreter, will sagen: dem E-Reader, das Dokument entsprechend einer personalisierten Schriftgröße zu rendern. Bei konstanter Größe der Sichtfläche ohne horizontale Scrollmöglichkeit führt das notwendig zu einer veränderten Umbruch und also zu einer geänderten Seitenzählung. Der gern genutzte \tit{Kindle} ist ein gutes Beispiel dafür.}:

Universitätsbibliotheken schaffen bestimmte Literatur 'nur' noch als E-Book oder als E-Paper an. Das ist ökonomisch sinnvoll. Es reduziert Lager- und Verwaltungskosten. Allerdings 'kaufen' die Bibliotheken oft nicht die elektronischen Kopien selbst, um diese von ihrem eigenen Server aus an die Leserinnen zu verteilen. Vielmehr erwerben sie von den Verlagen 'nur' das Recht, dass die über die Bibliothek authentifizierten Leserinnen sich eine Kopie aus dem Verlagsnetz downloaden dürfen.\footnote{Das ist im Übrigen ein technischer Grund dafür, warum ein solches Downloaden seitens der Verlage 'nur' aus dem Universtätsbibliotheksnetz heraus ermöglicht wird. Diese Hintergründe sind mir bei verschiedenen Veranstaltung zur Bibliotheksnutzung in Frankfurt und Darmstadt bestätitgt worden. Eine zitierfähige Beschreibung des Verfahrens steht -- laut Auskunft der Bibliothekarinnen (erfragt zuletzt am 26.02.2016) -- jedenfalls für Frankfurt nicht so einfach zur Verfügung.}

Damit wird die 'Verlässlichkeitskette' beschädigt: Verlage können nun einfach und 'unerwähnt' unter demselben Downloadlink die modifizierte Version eines Werkes anbieten. Textstellen könnten damit geändert, Seitenzahlen verschoben worden sein. Das muss nicht einmal absichtlich geschehen. Schon simple technische Fehler können solche Modifikationen zu bewirken.

Verschärft wird das Problem wird dadurch, dass man den zitierten 'E-Werken' nachträgliche Eingriffe nicht ansieht - ganz im Gegensatz zu gedruckten Werken, deren erkennbare Beschädigungen dann auch ihre Belegkraft diskreditieren.

So erschüttert die Nutzung von E-Quellen die Reproduzierbarkeit; sie untergräbt die für die Wissenschaftlichkeit konstitutive Verlässlichkeit der Forschungsliteratur: Dass E-Books und E-Papers 'stillschweigend' ersetzt und ihnen Änderungen nicht unbedingt angesehen werden können, macht es angreifbar, sie zu zitieren. Denn auch 'korrekt' ausgewiesene Zitate sind dann nicht mehr in einem letzten Sinne 'reproduzierbar' und also überprüfbar.\footnote{In diesem Zusammenhang wird erkennbar, dass Bibliotheken als staatliche Einrichtungen mit ihrer Aufgabe, den Bestand und die Beständigkeit der Forschungsliteratur zu horten, auch heute noch eine wissenschaftskonstitutive Funktion wahrnehmen.}

Noch deutlicher tritt dieser systematische Makel dort zu Tage, wo nicht einmal mehr Verlage hinter den über das Internet distribuierten E-Books stehen: Das \enquote{größte Risiko bei der Verwendung von im Internet generierten [\ldots] Materialien} liege - nach Ansicht des deutschen Standardwerks - darin, dass \enquote{[\ldots] nicht jeder Dateneingeber [\ldots] sich bzw. seine Dokumente nachhaltig vor Manipulationen schützen (könne)}\footcite[vgl.][85]{Theisen2013a}: \enquote{die Offenheit des Internet-Systems (erlaube) es, Nachrichten und \emph{Daten} zu verändern oder ganz \emph{zu verfälschen}}. Das bedeute, dass das Internet der \enquote{Flüchtigkeit des Mediums} und der Volatilität der URLs wegen \enquote{[...] nur im Ausnahmefall eine Nachprüfung der Informationen über einen längeren Zeitraum (zulasse)}\footcite[vgl.][S. 86f (herv.i.O)]{Theisen2013a}.

Und dennoch muss dieses Dilemma gelöst werden.\footnote{Glücklicherweise existieren bereits praktische Hinweise für einen guten Umgang mit 'Internetquellen': So enthält etwa das \emph{MLA Handbook for Writers of Research Papers} ein ganzes Kapitel zum \emph{Zitieren von Webpublikationen}. Die Conclusio ist hier, dass frau das besondere Format der Quelle durch ein Kürzel 'Web' in den bibliographischen Daten explizit macht, dass sie die URL des
zitierten Dokumentes hinzufügt und dass sie auch ihr je spezifisches Abrufdatum
in die bibliographischen Angaben integriert (\cite[vgl.][S. 28ff u. 181ff]{ModLanAss2009a}). Das deutsche Standardwerk geht noch einen Schritt weiter. Nachdem es die auf längere Zeit gesehen nur eingeschränkte Nachprüfbarkeit von Internetzitaten hervorgehoben hat, konstatiert es, dass \enquote{[...] elektronische Daten [\ldots] nachhaltig nachgewiesen werden (müssen), so dass der Leser (oder Prüfer) sie auch zu jedem späteren Zeitpunkt \emph{nachvollziehen} kann} (\cite[vgl.][S. 86f]{Theisen2013a}). Als Möglichkeiten für eine solche Verstetigung wird dann auf die \enquote{'Screenshot'-Technik} verwiesen (\cite[vgl.][80 u. 87]{Theisen2013a}) Ich werde so gleich eine verfeinerte Mixtur beider Ansätze als Lösung vorschlagen.}. Denn die E-Werke erfüllen fraglos das Kriterium der Zitierfähgikeit, sofern eben \enquote{[\ldots] alle Quellen und Sekundärmaterialien (zitierfähgig sind), die \emph{in irgendeiner Form} [\ldots] \emph{veröffentlicht} worden sind}.\footcite[vgl.][S. 160 (herv. K.R)]{Theisen2013a} Sie zu ignorieren, ist mithin keine Option.

Um einen potentiellen Mangel an Überprüfbarkeit transparent zu machen und
möglichst auszugleichen, bietet sich folgendes Verfahren an:

\begin{itemize}
  \item Bei gedruckten Werken, die sich eine Autorin über das normale Bibliotheks- resp. Verlagssystem beschafft und \tit{eigenhändig} ausgewertet hat, möge sie die bibliographischen Angaben im Literaturverzeichnis\footnote{unter Ausnutzung des Bib\LaTeX\ Tokens \ttt{note}} mit dem Schlagwort \emph{Print} markieren. Damit soll die Verfasserin nicht nur sagen, das Werk selbst 'physisch' eingesehen zu haben. Sie möge damit zudem markieren, die Angaben zur Quelle so genau spezifiziert zu haben, dass die Beschaffung über das normale Bibliothekssystem im Sinne der erwähnten Arbeitsteilung reproduzierbar ist.
  \item Werke, die eine Autorin über ein Netz in elektronischer Form eingesehen und ausgewertet hat, möge sie im Literaturverzeichnis\footnote{unter Ausnutzung des Bib\LaTeX\ Tokens \ttt{note}}  -- formatgemäß -- mit \emph{[BibWeb $|$ FreeWeb] / [PDF $|$ HTML $|$ \ldots]} markieren. Dabei stehe \emph{BibWeb} für ein durch eine Universitätsbibliothek bereitgestelltes Netz\footnote{das entweder physische Präsenz des Auswertenden in den Bibliotheksgebäuden oder die Nutzung eines entsprechenden VPN voraussetzt}, während \emph{FreeWeb} das frei zugängliche Internet meine. Bei Werken aus dem freien Internet möge eine Verfasserin die URL und das Datum in den Tokens \ttt{url} und \ttt{urldate} vermerken, unter der bzw. an dem sie den elektronischen Text eingesehen hat. Zudem sollte sie in beiden Fällen -- wo irgend möglich -- das eingesehene Werk als elektronische Kopie (PDF) sichern.\footnote{Es ist natürlich klar, dass auch die Autorin, die ihre volatilen Quellen zwecks späterem Nachweis als PDF sichert, diese PDF-Dateien selbst verändern kann - mit einem einfach Text-Editor und etwas PDF-Kenntnis. Mithin wäre ihr Nachweis angreifbar. Allerdings ist schon die bloße Möglichkeit, später die Originalform vorlegen zu können, ein Gewinn, dreht das doch die Beweislast um.} Diese darf sie aus Urheberrechtsgründen natürlich nicht frei weitergeben. Im Streitfall sollte es aber beruhigend sein, genau die Version unter juristisch kontrollierbaren Bedingungen selbst einsehbar machen zu können, die sie ausgewertet hat.
  \item In sehr selten Fällen wird eine Autorin die bibliographischen Angaben nicht verifizieren können, obwohl das Werk in der Forschungsliteratur durchgehend mit diesen Angaben zitiert wird. In solchen Fälle sollte sie es\footnote{unter Ausnutzung des Tokens \texttt{note}} mit \emph{[BibWeb $|$ FreeWeb] / REF} markieren. Nach menschlichem Ermessen wird ein solcher Fall aber nur im freien Web entstehen.
\end{itemize}

\emph{proScientia.ltx} unterstützt diese Verfahren von sich aus: Wenn man die o.a. Markierungen unter dem Token \ttt{note} in seine Bibtex-Bibliographie aufnimmt, werden sie an angmessener Stelle in die bibliographischen Angaben integriert. Die bereitgestellte \emph{JabRef}-Konfiguration berücksicht die Aktivierung des Tokens auch.

Und noch ein besonderer, aber nicht unüblicher Fall: Ich habe schon mehrfach auf Bib\LaTeX\ und damit auf ein \LaTeX-Paket verwiesen. Dieses Paket wird -- wie alle \TeX-Pakete -- über das \enquote{Comprehensive TEX Archive Network}, will sagen: \emph{CTAN} distribuiert.\footcite[vgl.][]{Ctan2022a} Nun ist CTAN ein so vielfach genutzter Service, dass er immer schon von sich aus Anfragen auf einen der vielen Spiegelserver umlenkt. Es macht also keinen Sinn, bei einem Zitat aus dem Handbuch in der Fußnote den tatsächlich genutzten Link\footnote{wie etwa \href{https://mirror.dogado.de/tex-archive/info/translations/biblatex/de/biblatex-de-Benutzerhandbuch.pdf}{https://mirror.dogado.de/tex-archive/info/translations/biblatex/de/biblatex-de-Benutzerhandbuch.pdf}} einzufügen. Der konkrete Spiegelserver könnte jederzeit ausgesetzt werden, das Handbuch an sich aber immer noch über einen andereren erreichbar sein. Sinnvoller ist es mithin, darauf zu vertrauen, dass CTAN dasselbe Benutzerhandbuch auch in Zukunft für meine Leserin bereitstellt, es ihr aber volatil über einen anderen Server ausliefert. Darum füge ich als URL bei CTAN nur den Link auf das Paket ein.\footcite[vgl][]{KiWeLe2021a}

%\bibliography{../bib/literature}
