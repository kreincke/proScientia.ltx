\section{Was der Zweck ist.}

Warum Wissenschaftlerinnen jede Übernahme als Zitat ausweisen sollen, begründet das deutschsprachige \enquote{Standardwerk} zum Thema \enquote{wissenschaftlichen Arbeiten}\footnote{\cite[vgl.][S. 159ff]{Theisen2013a}. Dass dieses Buch 2013 in der 16. Aufl. erschienen ist und mittlerweile in der 18. Aufl. bei angeboten wird (s. \lnkb{https://www.amazon.de/dp/3800663732/}{2022-02-13}, unterstreicht, das es wirklich ist, was sein Schmutzumschlag zu sein behauptet, nämlich ein \enquote{Standardwerk}.} legalistisch und funktional: einerseits verlange das Urhebergesetz solche Belege\footcite[vgl.][159]{Theisen2013a}, andererseits müsse \enquote{sichergestellt} sein, \enquote{[...] dass für wissenschaftliche Zwecke nur solches Material verwendet wird, das nachvollziehbar und damit auch \textbf{kontrollierbar} ist}.\footcite[vgl.][160]{Theisen2013a} Danach beschreibt dieses Buch verschiedene Nachweisformen\footnote{so den \enquote{Vollbeleg} (\cite[vgl.][S. 161f]{Theisen2013a}), den \enquote{Kurzbeleg}(\cite[vgl.][S. 163f]{Theisen2013a})}, diskutiert die Positionen der Nachweise im zitierenden Text\footcite[vgl.][S. 166ff]{Theisen2013a} und klassifiziert Zitate von ihrem Verhältnis zum Original her\footnote{so das \enquote{direkte} bzw. \enquote{wörtliche} Zitat (\cite[vgl.][S. 169ff]{Theisen2013a}), das \enquote{indirekte} Zitat (\cite[vgl.][S. 174ff]{Theisen2013a}), das \enquote{Sekundärzitat}  (\cite[vgl.][S. 177f]{Theisen2013a}) und das \enquote{Zitat im Zitat}(\cite[vgl.][179]{Theisen2013a})}. Und für all diese Aspekte hatte es vorab konstatiert, dass \enquote{die genaue Kenntnis und sorgfältige Berücksichtigung der unterschiedlichen \textbf{Zitatformen} [...] eine 'conditio sine qa non' (lat.: ein zwingendes Erfordernis) (sei) [...]}\footcite[vgl.][S. 159 {herv. u. übers. i.O.}]{Theisen2013a}.

Leider ist die Aussage, die 'Berücksichtung solcher Kenntnisse' sei eine 'condition sine qua non' vorderhand nur eine Behauptung. Als solche kann sie Zweck und Form des richitgen Zitierens nicht begründen. Hilfreicher ist da schon der Hinweis auf die 'Kontrollierbarkeit'. Ich sehe vier Gründe, Aussagen anderer zu zitieren\footnote{vgl. dazu \cite[][52]{ModLanAss2009a} und \cite[][187]{RueStaFra1980a}. Letztere beschreiben die Funktionen ähnlich, legen aber andere Schwer\-punk\-te: So läuft das, was ich als affirmatives Zitat bezeichnen, bei ihnen als 'Bestätigung wissenschaftlicher Thesen durch anerkannte Autoritäten oder Arbeiten', während das, was ich als 'konfrontatives Zitat' bezeichne, bei Ihnen nicht vorkommt.}:

\begin{enumerate}
  \item Jemand anderes hat einen Befund oder Ansatz geliefert, dessen Wahrheit, Gültigkeit oder Relevanz ich in Zukunft voraussetze. Ich referiere diesen Befund über \emph{affirmative Zitate}, baue meine Argumentation darauf auf und kann so einen Teil meiner Arbeit 'delegieren'. Mithin ist es in meinem Interesse, diese Vorarbeit 'kontrollierbar' zu zitieren, damit ich selbst in meiner Argumentation den Schritt weg von der bloßen Behauptung hin zum verifizierbaren Argument tue.
  \item Jemand anderes hat einen Befund oder Ansatz geliefert, dessen Wahrheit, Gültigkeit oder Relevanz ich bestreite. Über \emph{konfrontative Zitate} referiere ich zunächst diesen Befund, um ihn anschließend zu widerlegen. Wiederum ist es also in meinem Interesse, diese Vorarbeit 'kontrollierbar' zu zitieren, schärft doch die belegte Abgrenzung meinen eigenen Ansatz.
  \item Jemand anderes hat einen Begriff oder ein Wort benutzt, das ich
  übernehmen will. Damit delegiere ich die Arbeit der Definition an diesen
  anderen und referiere seine Ergebnisse über \emph{adaptive Zitate}. Erneut dient das überprüfbare Zitieren meiner eigenen Arbeit, kann ich bei Rückfragen auf den eigentlichen Schöpfer verweisen.
  \item Ich verweise - grosso modi - auf konkurrierende Positionen, andere
  Aspekte oder erweiterte Kontexte, um sie über
  \emph{abweisende Zitate} aus meinem Fokus auszugrenzen. Das genaue Zitieren macht hier überprüfbar, ob diese ausgegrenzten Aspekte 'abseitig' sind. Denn genau das habe ich ja dadurch behauptet, dass ich nur grosso modi auf sie verwiesen habe.
\end{enumerate}

Man sieht\footcite[vgl. dazu auch][hier werden im Abschnitt
'Wissenschaft' drei Funktionen aufgelistet. Das, was ich als 'affirmatives
Zitat' bezeichne, läuft - unter dem Schlagwort 'auf den Schultern von Riesen' -
als Redundanzreduktion, gepaart mit der Überprüfbarkeit. Außerden wird die Moral
ins Feld geführt]{Wikipedia2011a}: es ist bei jeder dieser Zitatfunktionen in
meinem Interesse, meine Quellen nicht nur 'irgendwie' anzugeben,
sondern es meinen Leserinnen leicht zu machen, sie wiederzufinden. Erschwere ich
es ihnen hingegen, schwäche ich meine Argumentation, schwäche ich mich. Denn
dann könnten sie bestenfalls über mich sagen: 'nun gut, er hat es zumindest
behauptet, aber ob's stimmt, wer weiß? - wir konnten es jedenfalls nicht
wirklich nachprüfen'.

Den argumentativen Funktionen des Zitats stehen seine inhaltlichen Formen gegenüber. Zu unterscheiden wäre zwischen \glqq{}wörtlichem\grqq{} oder \glqq{}nicht-wörtlichem\grqq{} Zitat, will sagen: der wortgetreuen und der \glqq{}[\ldots] sinngemäßen Übernahme oder Wiedergabe schriftlicher oder mündlicher Äußerungen anderer\grqq{}.\footcite[vgl.][S. 187f - ohne Frage, dieses ist ein sinngemäßes und kein wörtliches Zitat. Und es ist affirmativ]{RueStaFra1980a} Anderenorts werden sie als \emph{direktes} und \emph{indirektes Zitat} bezeichnet.\footcite[vgl.][]{WisArbOrgZitate} Persönlich würde ich hier noch feiner unterscheiden und diesen beiden Formen das \emph{begriffliche Zitat} zur Seite stellen:

\begin{description}
  \item[direktes Zitat] :- die wort- und zeichengetreue Wiedergabe (mindestens) eines Satzes (Aussage), ggfls. durch markierte Auslassungen 'konzentriert'. Der Zitator erhebt den Anspruch, exakt wiedergegeben zu haben.\footcite[vgl.][187f]{RueStaFra1980a} Auf die Quelle wird am Ende des in Anführungszeichen eingeschlossenen Textes direkt verwiesen.\footcite[vgl.][172]{Theisen2013a}
  \item[indirektes Zitat] :- eine Paraphrase, die (mindestens) einen Satz (Aussage) sinngemäß wiedergibt. Sie darf einzelne Termini oder Satzteile aus dem Original entnehmen, sofern sie diese mit Anführungszeichen markiert. Auf die Quelle wird am Ende der Paraphrase mit einer Fußnote verwiesen, in der die bibliographischen Daten mit \emph{vgl.} oder \emph{s.} eröffnet werden. Damit wird das direkte vom indirekten Zitat unterschieden.\footcite[vgl.][174]{Theisen2013a} Es signalisiert den Anspruch des Zitators, die Aussage als Ganzes sinngemäß, aber nicht wörtlich wiedergegeben zu haben.\footcite[vgl.][letzter Absatz aus Abschnitt 'Grenzen der Zitierpflicht']{Wikipedia2011a}
  \item[begriffliche Zitat] :- die wort- und zeichengetreue Übernahme eines Wortes bzw. einer Satzkonstituente als ein Begriff. Dieser übernommene Begriff wird in Anführungszeichen gesetzt, auf die Quelle wird unmittelbar nach dem Wort mit Hilfe von {\itshape vgl.} verwiesen. Der Zitator beansprucht damit, die Definition von jemand anderem übernommen zu haben, die Aussage, in die das Übernommene eingebettet ist, aber selbst zu verantworten.
\end{description}

Damit können wir die einfache Frage stellen, welche Zitatformen für welche
Zitatfunktionen dienlich sind:

\begin{center}
\begin{tabular}{|r||c|c|c|}
\hline
& {Direktes Zitat}
& {Indirektes Zitat}
& {Begriffliches Zitat}
\\
\hline \hline
\emph{affirmative Zitate}& \checkmark &  \checkmark & \\
\hline
\emph{konfrontative Zitate}&  \checkmark &  \checkmark & \\
\hline
\emph{adaptive Zitate}&  & \checkmark & \checkmark\\
\hline
\emph{abweisende Zitate}&  & \checkmark & \\
\hline
\end{tabular}
\end{center}

Bliebe noch zu klären, zu welchem Zweck solch ausgeklügelte Regeln befolgt werden sollen. Diese Frage wird besonders klar von dem anderen Standardwerk der Wissenschaftsgemeinde beantwortet, nämlich vom 'MLA Handbook for Writers of Research Papers', das sich selbst als 'The Authorative Guide' bezeichnet.\footcite[vgl.][Buchcover]{ModLanAss2009a} Dazu erläutert es zunächst, was Plagiate sind und was sie für die Forschung bedeuten.\footcite[vgl.][S. 52ff]{ModLanAss2009a} Dann erklärt es, wie Zitattexte korrekt erstellt werden\footcite[vgl.][92ff]{ModLanAss2009a}, um anschließend die Form des zugehörigen Beleges\footcite[vgl.][S. 126ff]{ModLanAss2009a} und die dafür konstitutive \glqq{}List of Works Cited\grqq{}, die Literaturliste zu beschreiben.\footcite[vgl.][126ff]{ModLanAss2009a} Und dabei formuliert es einen beeindruckenden Anspruch:

\begin{quote}\glqq{}They [the responsible writers; KR] specify when they refer to another author's ideas, facts, and words, whether they want to agree with, object to, or analyze the source. This kind of documentation not only recognizes the work writers do; it also tends to discourage the circulation of error, by inviting readers to determine for themselves wether a reference to another text presents a reasonable account of what the text says.\grqq{}\footcite[][52]{ModLanAss2009a}
\end{quote}

Zentral ist hier, dass Leserinnen dazu \textit{eingeladen} (und nicht: daran gehindert) werden sollen, Aussagen anderer Autorinnen, die im gerade gelesenen Text zitiert worden sind, \textit{eigenhändig zu überprüfen}, und zwar nicht nur, ob die Aussagen korrekt wiedergegeben sind (das ist 'nur' eine notwendige Voraussetzung), sondern ob sie in die Argumentation auch \textit{valide eingebunden worden sind} und diese stützen. Zu dieser Forderung an Autorinnen sagt das Handbuch schlicht:

\begin{quote} \glqq{}Plagiarists undermine these important public value. Once detected, plagiarism in a work provokes skepticism and even outrage among readers, whose trust in the author has been broken.\grqq{}\footcite[][52f]{ModLanAss2009a} \end{quote}

Ein solcher Schaden - so das Handbuch - entstehe sogar durch 'unbeabsichtigte Plagiate'.\footcite[vgl.][55 - im Original \glqq{}unintenional plagiarism\grqq{}]{ModLanAss2009a} Und diese können leichter 'entstehen', als eine unbedarfte Autorin anzunehmen geneigt ist. Denn so gelte z.B.:

\begin{quote}\glqq{}Presenting an author's wording without marking it as quotation is plagiarism, even if you cite the source.\grqq{}\footcite[][55 (herv.KR.)]{ModLanAss2009a} \end{quote}

Der tiefere Grund für diese 'Dippelschisserei' liegt darin, dass es zum Wesen der Wissenschaft gehöre, an Vorarbeiten anzuknüpfen: Der Zweck eines Forschungspapieres \glqq{}[\ldots] is to synthesize previous research and scholarship with your ideas on the subject\grqq{}. Und wenn das 'Borgen' intentional schon dazugehöre, dann dürfe \glqq{}[\ldots] the material you borrow [\ldots] not be presented as if it were your own creation\grqq{}.\footcite[vgl.][55]{ModLanAss2009a} Klar, dass das unmarkierte Zitat diese Regel verletzt. Denn wie sollte aus der bloßen Quellenangabe geschlossen werden können, welche Wörter übernommen und welche eigene Zutat sind? Eigentlich also kaum noch erwähnenswert, weil implizit unabdingbar, ist dann noch die folgende ergänzende Regel

\begin{quote} \glqq{}[\ldots] you must document everything that you borrow - not only direct quotations and paraphrases but also information and ideas.\grqq{}\footcite[][52f]{ModLanAss2009a} \end{quote}

Vom Zweck her decken sich also meine Wünsche und die Ansprüche des MLA Handbuches. Wir unterscheiden uns, wenn es um die Form geht. Dazu trägt insbesondere eine zentrale Anweisung des MLA Handbuches bei:

\begin{quote} \glqq{}[\ldots] A citation in MLA style contains only enough information to enable the readers to find the source in the works-cited list.\grqq{}\footcite[][127]{ModLanAss2009a} \end{quote}

Das hat radikale Konsequenzen: Werde in einem 'Erzähltext' beispielsweise der Name einer Autorin erwähnt, von der nur ein Werk zitiert wird, dann reiche es, im Erzähltext nach dem Zitat die bloße Seitenzahl anzugeben. Erst wenn es mehrere Werke seien, müsse zusätzlich zur Seitenzahl ein so gekürzter Titel im laufenden Text eingefügt werden, dass das Werk in der Literaturliste wiedergefunden werden könne.\footnote{\cite[vgl.][127]{ModLanAss2009a}. Fairerweise erwähnt das Handbuch aber, der \glqq{}[\ldots] MLA is not the only way to document sources\grqq{} (\cite[vgl.][127]{ModLanAss2009a}) Eine Alternative sei (etwa) der 'APA style', bei dem im laufenden Text Autorin, Jahr und Seitenzahl angegeben werden und als Muster in das Literaturverzeichnis
verweisen. (\cite[vgl.][S. 127f]{ModLanAss2009a})}

Man darf mithin sagen, der MLA-Stil ist konsequent minimalistisch, also schreiberinnenfreundlich, nicht leserinnenfreundlich. Das erschwert es mir, ihn zu verwenden:

\begin{itemize}
  \item Zum ersten wird der Lesefluss, das 'gleichmäßige' Gleiten des Blickes
  über die Zeilen durch oft eben doch längliche Zitatbelege unterbrochen.
  \item Zum zweiten muss ich mir die Informationen aus dem Kontext
  'zusammenklauben', wenn ich ein Zitat überprüfen will. Wo stand noch gleich
  der Autorinnenname? Welche Seitenangabe bezog sich jetzt grad noch auf sein Werk?
  Mich lädt diese Art nicht ein, das Vorgetragene zu überprüfen.
  \item Und zum dritten und entscheidenden: In diesem Stil können mir Autorinnen
  nicht nebenbei die Forschungssgeschichte vermitteln. Dem steht der Minimalismus entgegen, der die forschungsgeschichtlichen Zusatzhinweise und Markanten wie Verlag, Auflage oder Jahr, wie Name der Zeitschrift oder der Serie etc. etc. in den Anmerkungen einfach beiseite lassen muss.
\end{itemize}

Der inhaltlich sicher gute Artikel 'Intellectualism as Cognitive Science' weist genau dieses unruhige Lese-Bild auf, das zwar wissenschaftlich korrekt ist, aber rezeptiv stolpern lässt: Die Zitate werden im laufenden Text innerhalb von Klammern nach dem Schema 'Autorin, Jahr, Seite' belegt.\footcite[vgl.][25]{RotCum2011a} Dabei unterbrechen die Klammern den Lesefluss. Und die Leserinnen, die den Titel des zitierten Werkes und damit die engste Zusammenfassung des Inhalts kennenlernen wollen, müssen ins Literaturverzeichnis blättern.\footnote{\cite[vgl.][S. 38f]{RotCum2011a}. Andere Werke referieren sogar nur noch über 'Autorin und Jahr' und verzichten ganz auf Seitenzahlen. Damit wäre die Überprüfbarkeit nicht nur stilistisch erschwert, sondern gänzlich verloren gegangen (\cite[vgl. z.B.][151]{Bechtel2011a})}

Der Vorteil des MLA-Stiles ist zugleich sein Nachteil: Er behandelt seine Lese\-rin\-nen auf Augenhöhe. Er geht unter der Hand davon aus, dass jede Leserin die (Mehrheit der) zitierten Werke im Prinzip schon kennt. Minimalistisch diskutiert die kundige Autorin mit der lesenden Expertin. Das Problem ist nur: Nicht alle Leserinnen sind immer schon Expertinnen. Und warum sollte eine Wissenschaftlerin es ihrer Leserin nicht erleichtern, selbst zur Expertin zu werden?

Deshalb ziehe ich den (alt)philologisch/philosophischen Zitierstil vor. Er hat mir die Rezpetion erleichtert, nicht nur die eines Werkes, sondern die des Faches selbst. Und in diesem Sinne will ich auch meinen Leserinnen dienen.


%\bibliography{../bib/literature}
