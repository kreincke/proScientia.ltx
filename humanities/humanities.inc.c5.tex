% This file is part proScientia.ltx/humanities.
% (c) 2022 K. Reincke (https://github.com/kreincke/proScientia.ltx/humanities)
% Unlike the other files of proScientia, all files under proScientia/humanities
% are licensed under the terms of the creative commons license CC-BY-SA-4.0
% (= https://creativecommons.org/licenses/by-sa/4.0/)

\section{Zusammenfassung}

Fassen wir zusammen: Verglichen mit dem numerischen Verweisen oder kryptischen Schlüsselreferenzen innerhalb des Lesetextes, ja selbst verglichen mit stark verkürzendem Autor-Jahr-Schema bieten uns Bib\LaTeX\ und \tit{biblatex-dw} -- entsprechend konfiguriert - eine lese- und lernbegünstigende Alternative: Der Anmerkungsapparat bedient seine immanente Aufgabe, Zitate zu belegen. Und zugleich liefert er eine forschungshistorische Zuarbeit. Er breitet vor dem Leser vertrackte Aspekte der Wissenschaftsgeschichte aus und reicht damit die schmerzliche Detailarbeit seiner Autorin uneigenützig an die Leserin weiter. Wissen ist hier nicht mehr Macht, Gelehrsamkeit nicht mehr Klientel stabilisierendes Herrschaftswissen, sondern schlichter \emph{Dienst an der Kundin}.
