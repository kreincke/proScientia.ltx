% This file is part proScientia.ltx/humanities.
% (c) 2022 K. Reincke (https://github.com/kreincke/proScientia.ltx/humanities)
% Unlike the other files of proScientia, all files under proScientia/humanities
% are licensed under the terms of the creative commons license CC-BY-SA-4.0
% (= https://creativecommons.org/licenses/by-sa/4.0/)

\section{Wie es umgesetzt wird.}

Mein Wunsch über all die Jahre war, dass der altphilologisch geisteswissenschaftliche Schreib- und Argumentierstil \LaTeX-like ermöglicht wird: Ein simpler Befehl für den Zitatbeleg, und der Rest sollte sich von allein
ergeben, gerne mittels Zuladung von Paketen gesteuert, über Konfigurationen
verfeinert und mit Bib\TeX\footcite[vgl.][]{BibtexOrgDe} oder Bib\LaTeX\footcite[vgl.][]{BibLaTeX2022a} umgesetzt. Das Berücksichtigung der fitzeligen Kleinigkeiten eines korrekten bibliographischen Nachweises jedoch wollte automatisiert sehen.

Mit einem modifizierten \emph{Jurabib}\footcite[vgl.][]{Berger2004a} und \emph{KOMA-Script}\footcite[vgl.][]{Kohm2008a} war ich dem schon Anfang 2000 nahe gekommen. Aber eben nicht perfekt. Trotzdem hatte ich die Modifikationen unter dem Namen \emph{mycsrf} in einem Repository samt IDE zur Verfügung gestellt\footcite[vgl.][]{Reincke2021a} und sogar ein passendes Handbuch dazu offeriert.\footcite[vgl.][]{Reincke2018a}

Nun aber ist all dies nicht mehr notwendig.\footnote{eigentlich schon seit spätestens 2016, nur habe ich die bessere Lösung leider erst 2018 entdeckt} Es geht viel einfacher und schlanker, mit \LaTeX, Bib\LaTeX\footcite[vgl.][]{BibLaTeX2022a},  Biber\footcite[vgl.][]{Biber2022a} und \emph{biblatex-dw}\footcite[vgl.][]{Waßenhoven2016a}. Möchte eine Autorin verstehen, wie das zusammenwirkt, muss sie in erster Linie \LaTeX\ kennen und können. Dann kann sie dazu die Quellen dieses Artikels untersuchen oder - noch einfacher - die Templatelösung \emph{proScientia.ltx} auschecken. In aller Kürze skizziert, funktioniert es so:

Zuerst wird im Header des eigenen \LaTeX-Textes das \emph{biber}-Package und der \emph{authortitle-dw}-Stil eingebunden.\footnote{Dieser ist das, was wir meinen, wenn wir von \emph{biblatex-dw} sprechen.} Dann muss eine dazu passende Konfiguration eingebunden werden, die die Details des Erscheinungsbildes festelegt:

\small
\begin{verbatim}
\usepackage[
  backend=biber,
  style=authortitle-dw,
  sortlocale=auto,
]{biblatex}
\ProvidesFile{biblatex.cfg}
% This file is part of proscientia.ltx
% (c) 2022 Karsten Reincke (https://github.com/kreincke/proscientia.ltx)
% It is distributed under the terms of the creative commons license
% CC-BY-4.0 (= https://creativecommons.org/licenses/by/4.0/)


\ExecuteBibliographyOptions{
  %%%%%%%%%%%%%%%%%%%%%%%%%%%%%%%%%%%%%%%%%%%%
  % configurations offered by biblatex itself
  % ------------------------------------------
  maxnames=5,  % Truncate author list after 5 authors ...
  minnames=3,  % ... But display at least 3 authors
  autocite=inline,
  hyperref=true,  % Use hyperref package (should be automatically detected, though)
  backref=true,  % Back references from bibliography page to each encounter
  backrefstyle=two,  % Combine back refs if on two consecutive pages
  isbn=true,  % (Dont) print ISBN, ISSN numbers
  autolang=hyphen,
  % track 'reused' csquotes
  citetracker=constrict, %
  loccittracker=constrict, % discriminate different pages (a.a.O) versues same page (ebda)
  opcittracker=constrict,
  idemtracker=constrict,
  ibidtracker=constrict,
  pagetracker=true,
  %%%%%%%%%%%%%%%%%%%%%%%%%%%%%%%%%%%%%%%%%%%%
  % configurations offered by authortitle-dw
  % ------------------------------------------
  annotation =true,
  namefont=normal,
  firstnamefont=normal,
  idemfont=italic,
  ibidemfont=italic,
  idembib=true, % cluster the books of the same author in bib
  idembibformat=idem, % indicate the same author by ders/dies.
  editorstring=brackets, % parens=(Hrsg.) | brackets=[Hrsg.] | normal = , Hrsg.
  nopublisher=false, % insert publisher into bib data
  editionstring=true, % allow strings in the edition field
  % ZNAME VOL (YEAR) Nr. YOURNALNUMBER
  journalnumber=afteryear, %
  %journumstring=h.
  series=afteryear,
  seriesformat=parens,
  addyear=true, %insert year after titel in 'shorttitle' hints => no year in shorttitle
  firstfull=true, % frist quote complete
  edstringincitations=false, %editor and translator only in the first
  citepages=separate, % vollzitat erst mit seiten, dann 'hier: S. 12'
}
% Put your definitions here.

% refine seriesformat by adding a prefix inside of the parens
\renewcommand*{\seriespunct}{=\addspace}

% by default Biblatex-dw uses Autor1/Autor2 in cites
% these redefinitions overwrite that behaviour by
% duplicating the style used for the bibliography
% (S. p. 35 in the German biblatex-dw handbook )
\renewcommand*{\citemultinamedelim}{\addcomma\space}
\renewcommand*{\citefinalnamedelim}{%
\ifnum\value{liststop}>2 \finalandcomma\fi
\addspace\bibstring{and}\space}%
\renewcommand*{\citerevsdnamedelim}{\addspace}

% biblatex-dw does not print 'ders.' / 'dies.' in a row of cites quoting
% the same book. Additionally, it does not know the diffrence between the
% same page of of the same work and a different page of the same work
% quoted before.
%
% As soon as biblatex 3.15 is offered by UBUNTU
% use \bibncpstring[\mkibid]{ibidem} instaed of inserting the
% German string literally as this hack does:
\renewbibmacro*{cite:ibid}{%
{\ifthenelse{\ifloccit}
  {\printtext[bibhyperref]{\usebibmacro{cite:idem} \mkibid{ebda}}%
   \global\booltrue{cbx:loccit}}
  {\printtext[bibhyperref]{\usebibmacro{cite:idem} \mkibid{a.a.O, }}}
}
}%
\DefineBibliographyStrings{english}{%
  urlseen = {reviewed},
}
\DefineBibliographyStrings{german}{%
  urlseen = {heruntergeladen am},
}
\endinput

\end{verbatim}
\small

\emph{proScientia.ltx} stellt eine solche Konfigurationsdatei so bereit, wie sie zur Erzeugung dieses Textes verwendet worden ist. Jedenfalls harmoniert so auch diese neue Lösung von \emph{Dirk Waßenhoven} wunderbar mit
\emph{KOMA-Script}.\footcite[vgl.][]{Kohm2008a}

Schließlich muss im Header mittels des Befehls \texttt{\textbackslash{}addbibresource\{xyz.bib\}} noch festgelegt werden, welche Bibliotheksdateien ausgewertet werden sollen.

Hat die Autorin dann einen wissenschaftlichen Text mit Fuß- oder Endnoten erstellt, kann er leicht als PDF kompiliert werden:

In der Bibliographiedatei wurden die bibliographischen Daten der zitierten Werke ja zu Gruppen zusammengefasst und je Werk einem Identifier versehen. Im zitierenden Text schlägt das \emph{footnote}-Tag die Brücke: \texttt{\textbackslash{}footcite[vgl.][25]\{Kohm2008a\}}. Und zuletzt bringt die Aufrufsequenz \texttt{latex biber latex latex} die Dinge zusammen. \emph{proScientia.ltx} stellt für einen automatisierten Ablauf ein \emph{Makefile} bereit.
