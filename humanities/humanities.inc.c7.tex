\section{Die Lust des freien Austauschs}

Bliebe noch zu sagen, dass mein Anteil an der skizzierten Lösung im Lesen und Anwenden der Vorarbeit anderer besteht: Auf die Arbeiten von \tit{Dominik Waßenhoven} und auf Bib\LaTeX\ habe ich bereits verwiesen. Außerdem zehre ich immer noch vom \LaTeX-Begleiter\footcite[vgl.][S. 741ff]{MitGoo2005a} und von der Einführung von Helmut Kopka\footcite[vgl.][]{Kopka2000a}: Ohne deren freigiebige Darstellung wüsste und könnte ich heute nicht das, was ich weiß und kann.

In ähnlichem Sinne befördert hat mich die Tatsache, dass wir es hier mit freier Software zu tun haben: \LaTeX\ ist frei, ebenso Bib\LaTeX, \tit{Jurabib},  \tit{biblatex-dw}, \tit{Koma-Script}, \tit{Texlipse}, das \LaTeX-Plugin für \tit{Eclipse}, der Editor \tit{atom}  und eben \tit{JabRef}. Es ist also eine Frage des Anstands, dass auch ich \textit{proscientia.ltx} frei zugänglich mache:

Das Framework \tit{proscientia.ltx}, das die Suche und Evaluation von Sekundärliteratur unterstützen, die Pflege der bibliographischen Daten vereinfachen, die Erstellung dazu passender ’Abstracts’ und ’Extracts’ ermöglichen und das schließlich auch das Schreiben der eigentlichen Arbeit erleichteren soll, ist unter der \tit{Creative Commons 4.0 License} veröffentlicht.\footnote{s. \lnka{http://github.com/kreincke/proscientia.ltx/}}

Das Dokument jedoch, was Sie gerade lesen, ist -- obwohl auch Teil des Paketes -- davon unabhängig unter der \textit{Creative Commons Attribution-ShareAlike 4.0 Germany License} veröffentlicht.

Damit ist die Sache ganz einfach: auf dem Framework \tit{proscientia.ltx} können Sie ihre eigenen Arbeiten im Rahmen der \tit{CC-BY-4.0}-Lizenz aufsetzen und vertreiben. Wenn sie jedoch an diesem Text über \textit{Über hilfreiche Anmerkungsapparate} weiterarbeiten wollen, geben Sie Ergebnis bitte unter derselben Lizenz weiter.
