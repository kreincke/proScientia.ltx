% This file is part proScientia.ltx/humanities.
% (c) 2022 K. Reincke (https://github.com/kreincke/proScientia.ltx/humanities)
% Unlike the other files of proScientia, all files under proScientia/humanities
% are licensed under the terms of the creative commons license CC-BY-SA-4.0
% (= https://creativecommons.org/licenses/by-sa/4.0/)

\section{Die Lust am Gendern}

Wikipedia rechnet den \enquote{geschlechterbewussten Sprachgebrauch} dem \emph{Gendern} im Allgemeinen zu\footcite[vgl.][]{Wikipedia2022a} und definiert die \enquote{geschlechtergerechte Sprache} an sich als einen Sprachgebrauch, \enquote{der in Bezug auf Personenbezeichnungen [...] die Gleichstellung der Geschlechter in gesprochener und geschriebener Sprache zum Ausdruck bringen will}.\footcite[vgl.][]{Wikipedia2022b}  Bib\LaTeX\ bietet dazu -- anders als noch Bib\TeX -- endlich auch eine automatisierte Version; ich hatte das Eingangs demonstriert\footnote{s.S. \pageref{Gender}}:

Bib\LaTeX\ erlaubt es nämlich, jeden Titel mit dem Tag \ttt{gender} auszustatten. Das Handbuch definiert die Semantik des Feldes als \enquote{das Geschlecht des Autors oder das Geschlecht des Herausgebers, wenn es keinen Autor gibt} und stellt folgende syntaktischen Verfeinerungen über den Wert des Feldes bereit: \enquote{sf (femininer Singular, ein einzelner weiblicher Name), sm (maskuliner Singular, ein einzelner männlicher Name), sn (Neutrum Singular, ein einzelner neutraler Name), pf (femininer Plural, mehrere weibliche Namen), pm (maskuliner Plural, mehrere männliche Namen), pn (Neutrum Plural, mehrere neutrale Namen), pp (Plural, mehrere Namen unterschiedlichen Geschlechts)}.\footcite[vgl.][27]{BibLaTeX2021a} Allerdings benötigt frau dazu noch einen Bibliographiestil, der diese Daten auch auswertet. \emph{biblatex-dw} tut das exezellent. So weit, so klar.

Nun stellt sich jedoch die Frage, mit welchem Geschlechter-Kürzel Internetseiten etc. ausgezeichnet werden sollen, die direkt keinen Autor ausweisen, aber sicher einen oder mehrere haben. Naheliegend wäre es, sie als singuläres Neutrum zu markieren (= \ttt{sn}). Das führt aber zu sprachlich hässlichen Auswüchsen: alle Wikipediaseiten würden im Literaturverzeichnis hintereinandergereiht und mit \emph{dass.} für \emph{dasselbe} eröffnet.

Meine bisher beste Lösung dafür ist die:

\begin{itemize}
  \item Internetseiten ohne jede Autorinnenangabe zeichne ich je als singuläres Femininum (= \ttt{sf}) aus.
  \item Internetseiten, von denen ich weiß, dass sie von einer Gruppe erarbeitet werden, dessen Zusammensetzung ich nicht kenne, markiere ich als (= \ttt{pp})
  \item Internetseiten, die von einer Organisation oder Firma bereitgestellt werden, zeichne ich ebenfalls als singuläres Femininum (= \ttt{sf}) aus.
\end{itemize}

Bei all diesen Varianten verwende ich \ttt{anon.} für \emph{anonymus}\footnote{respektive \emph{anonyma}} als Autorinnenname und hänge dem -- wo möglich -- in eckigen Klammern den Namen der Organisation oder Firma an. Jeder Wikipedia-Eintrag hätte also bei mir also die Autorin \ttt{anon. [Wikipedia]}. So liest sich das implizit generalisierte generische Femininum in deutschen Texten flüssig, in englischen ist es eh egal.

Allerdings entsteht damit eine nächste Herausforderung: Nun werden alle Werke von anonymen Autorinnen genau der einen \emph{anonyma} zugeordnet und im Literaturverzeichnis hintereinander aufgelistet, jeweils eröffnet mit \emph{dies.} Das heißt, dass alle Internetseiten etc. implizit derselben Autorin zugeschrieben werden. Das ist unschön. Es lässt sich aber wenigestens dadurch entschärfen, dass frau im bib-File nach dem Tag \ttt{author} auch noch das Tag \ttt{sortname = \{Wikipedia\}} verwendet. So würden wenigstens alle Wikipedia-Seiten etc. jeweils gesondert geclustert.
