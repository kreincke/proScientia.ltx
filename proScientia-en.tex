% This file is part of proScientia.ltx
% (c) 2022 Karsten Reincke (https://github.com/kreincke/proScientia.ltx)
% It is distributed under the terms of the creative commons license
% CC-BY-4.0 (= https://creativecommons.org/licenses/by/4.0/)

\documentclass[
  DIV=calc,
  BCOR=5mm,
  11pt,
  headings=small,
  oneside,
  abstract=true,
  toc=bib,
  ngerman,english]{scrbook}

%%% (1) general configurations %%%
\usepackage[utf8]{inputenc}
\usepackage{a4}
\usepackage[ngerman,english]{babel}

\usepackage[
  backend=biber,
  style=authortitle-dw,
  sortlocale=auto,
]{biblatex}
\input{cfg/inc.cfg-biber-en.tex}

\addbibresource{bib/lit.verify.bib}
\addbibresource{bib/lit.main.bib}

% package for improving the grey value and the line feed handling
\usepackage{microtype}

%language specific quoting signs
\usepackage[
  style=german,
  autostyle=true,
]{csquotes}

% language specific hyphenation
\input{cfg/inc.babelhyphenations.tex}

%%% (3) layout page configuration %%%

% select the visible parts of a page
% S.31: { plain|empty|headings|myheadings }
\pagestyle{headings}

% select the wished style of page-numbering
% S.32: { arabic,roman,Roman,alph,Alph }
\pagenumbering{arabic}
\setcounter{page}{1}

% select the wished distances using the general setlength order:
% S.34 { baselineskip| parskip | parindent }
% - general no indent for paragraphs
\setlength{\parindent}{0pt}
\setlength{\parskip}{1.2ex plus 0.2ex minus 0.2ex}


%- start(footnote-configuration)

\deffootnote[1.5em]{1.5em}{1.5em}{\textsuperscript{\thefootnotemark)\ }}

%for using label as nameref
\usepackage{nameref}

%integrate nomenclature
% This file originally comes from 'lrt4cs' [(c) 2020 Karsten Reincke,
% https://www.fodina.de/lrt4cs] that is distributed under the terms
% of CC-BY-3.0-DE (= https://creativecommons.org/licenses/by/3.0/)

\usepackage[intoc]{nomencl}
\let\abbr\nomenclature
% Deutsche Überschrift
%\renewcommand{\nomname}{Abbreviations}
\renewcommand{\nomname}{Abkürzungen}

\setlength{\nomlabelwidth}{.20\hsize}
\renewcommand{\nomlabel}[1]{#1 \dotfill}
% reduce the line distance
\setlength{\nomitemsep}{-\parsep}
\makenomenclature


% Hyperlinks
\usepackage{hyperref}
\hypersetup{bookmarks=true,breaklinks=true,colorlinks=true,citecolor=blue,draft=false}

\begin{document}

%% use all entries of the bliography
\nocite{*}

\titlehead{Classification}
\subject{Release \input{cfg/inc.rel.tex}}
\title{Title}
\subtitle{Subtitle}
\author{Author% This file originally comes from 'lrt4cs' [(c) 2020 Karsten Reincke,
% https://www.fodina.de/lrt4cs] that is distributed under the terms
% of CC-BY-3.0-DE (= https://creativecommons.org/licenses/by/3.0/)


\footnote{\textbf{This text is distributed under the terms of the XYZ license.} Please insert your conditions here - for example by choosing a Creative Commons License \texttt{https://creativecommons.org/}. Alternatively, one often sees \emph{All rights reserved}. Howsoever, due to the fact, that \textit{lrt4cs} is licensed under the \texttt{CC-BY-3.0-DE}, you must also point out that you've used it to created you own work. Such a license fulfilling sentence could be:\newline
{\small\itshape Created by using the package \texttt{\textbf{lrt4cs}} \copyright{} 2020 K. Reincke (\href{https://fodina.de/lrt4cs}{https://fodina.de/lrt4cs}) that is CC-BY-3.0-DE licensed (\href{https://creativecommons.org/licenses/by/3.0/}{https://creativecommons.org/licenses/by/3.0/})}}
}

\maketitle

\chapter{Introduction}

\section{Verifying the Quotation Technique}


\begin{itemize}

  \item \enquote{Quotation with an \enquote{embedded quotation}}

  \item An English sentence \foreignquote{english}{mit einer deutsche Phrase} as an embedded quotation. Expected result: English quotation marks because part of a surrounding English sentence.

  \item Separated German quotation:
  \begin{quote}
    \foreignquote{german}{Dieses ist ein deutscher Satz, der ein deutsch-sprachiges eigenständiges Zitat sein soll.}
  \end{quote}
  Expected Result: German quotation marks, because a separated part.
\end{itemize}

\section{Verifying the Literature Management}
\begin{itemize}
  \item book-1 (mentioned the first time)\footcite[vgl.][15]{KantKdrV1974}
  \item idem ibid.\footcite[vgl.][15]{KantKdrV1974}
  \item idem loc. cit.\footcite[vgl.][23]{KantKdrV1974}
  \item book-2 (mentioned  the first time)\footcite[vgl.][15]{KantKdU1974}
  \item book-2 short-title (mentioned the second time) \footcite[vgl.][15]{KantKdrV1974}
\end{itemize}

\chapter{Snippet Demo}
\begin{quote}\itshape
In this chapter we discuss \ldots
\end{quote}

\section{Snippet}
% This file originally comes from 'lrt4cs' [(c) 2020 Karsten Reincke,
% https://www.fodina.de/lrt4cs] that is distributed under the terms
% of CC-BY-3.0-DE (= https://creativecommons.org/licenses/by/3.0/)

%% use all entries of the bibliography

\subsection{snippet section 1}
\subsection{snippet section 2}
\subsection{snippet section 3}




\input{bib/ncl.abbrevs-en.tex}
% This file originally comes from 'lrt4cs' [(c) 2020 Karsten Reincke,
% https://www.fodina.de/lrt4cs] that is distributed under the terms
% of CC-BY-3.0-DE (= https://creativecommons.org/licenses/by/3.0/)

\abbr[afda]{AfdA}{Anzeiger für deutsches Altertum}
\abbr[zfda]{ZfdA}{Zeitschrift für deutsches Altertum und deutsche Literatur [ISSN: 00442518]}
\abbr[zfaw]{}{Zeitschrift für Allgemeine Wissenschaftstheorie / Journal for General Philosophy of Science [ISSN: 0044-2216]}

\printnomenclature
\printbibliography

\end{document}
