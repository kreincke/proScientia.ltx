% This file is part of proScientia.ltx
% (c) 2022 Karsten Reincke (https://github.com/kreincke/proScientia.ltx)
% It is distributed under the terms of the creative commons license
% CC-BY-4.0 (= https://creativecommons.org/licenses/by/4.0/)

\documentclass[
  DIV=calc,
  BCOR=5mm,
  11pt,
  headings=small,
  oneside,
  abstract=true,
  toc=bib,
  english,ngerman]{scrbook}

%%% (1) general configurations %%%
\usepackage[utf8]{inputenc}
\usepackage{a4}
\usepackage[english,ngerman]{babel}

\usepackage[
  backend=biber,
  style=authortitle-dw,
  sortlocale=auto,
]{biblatex}
\input{cfg/inc.cfg-biber-de.tex}

\addbibresource{bib/lit.verify.bib}
\addbibresource{bib/lit.main.bib}

% package for improving the grey value and the line feed handling
\usepackage{microtype}

%language specific quoting signs
\usepackage[
  style=german,
  autostyle=true,
]{csquotes}

% language specific hyphenation
\input{cfg/inc.babelhyphenations.tex}

%%% (3) layout page configuration %%%

% select the visible parts of a page
% S.31: { plain|empty|headings|myheadings }
\pagestyle{headings}

% select the wished style of page-numbering
% S.32: { arabic,roman,Roman,alph,Alph }
\pagenumbering{arabic}
\setcounter{page}{1}

% select the wished distances using the general setlength order:
% S.34 { baselineskip| parskip | parindent }
% - general no indent for paragraphs
\setlength{\parindent}{0pt}
\setlength{\parskip}{1.2ex plus 0.2ex minus 0.2ex}


%- start(footnote-configuration)

\deffootnote[1.5em]{1.5em}{1.5em}{\textsuperscript{\thefootnotemark)\ }}

%for using label as nameref
\usepackage{nameref}

%integrate nomenclature
\input{cfg/inc.cfg-ncl-de.tex}

% Hyperlinks
\usepackage{hyperref}
\hypersetup{bookmarks=true,breaklinks=true,colorlinks=true,citecolor=blue,draft=false}

\begin{document}

%% use all entries of the bliography
\nocite{*}

\titlehead{Klassifikation}
\subject{Release \input{cfg/inc.rel.tex}}
\title{Titel}
\subtitle{Untertitel}
\author{Autor% This file originally comes from 'lrt4cs' [(c) 2020 Karsten Reincke,
% https://www.fodina.de/lrt4cs] that is distributed under the terms
% of CC-BY-3.0-DE (= https://creativecommons.org/licenses/by/3.0/)


\footnote{\textbf{Dieser Text wird unter der XYZ Lizenz veröffentlicht.}
Hier können Bedingungen stehen, unter denen Sie Ihren Text weitergeben.
Gute Kandidaten wären z.B, die Creative Commons Lizenzen
\texttt{https://creativecommons.org/}. Traditionell ist auch die Formel:
\emph{Alle Rechte vorbehalten. Die Verwendung von Text und Bildern, auch
auszugsweise, bedürfen der schriftlichen Zustimmung.}. Aber wie auch immer: da \textit[lrt4cs] unter der Lizenz \texttt{CC BY 3.0 DE} veröffentlicht worden ist, müssen Sie auf dessen Verwendung hinweisen. Ein entsprechender lize-erfüllender Satz könnte z.B. dieser sein:
\newline
{\small \itshape Erarbeitet mit dem Paket \texttt{\textbf{lrt4cs}} \copyright{} 2020 K. Reincke (\href{https://fodina.de/lrt4cs}{https://fodina.de/lrt4cs}), das CC-BY-3.0-DE ( \href{https://creativecommons.org/licenses/by/3.0/}{https://creativecommons.org/licenses/by/3.0/}) lizenziert wurde.}}
}

\maketitle

\chapter{Einleitung}

\section{Verifikation: Zitatmarkierung}


\begin{itemize}

  \item \enquote{Zitat mit \enquote{eingebettetem Zitat}}

  \item Deutscher Satz \foreignquote{german}{with embedded foreign phrase}
  als eingebettetes Zitat. Erwartetes Resultat: Deutsche Anführungszeichen,
  weil Teil in einem deutsche Satz.

  \item Autonomes englischsprachiges Zitat:
  \begin{quote}
    \foreignquote{english}{This shall be an English written paragraph containing
    a set of sentences which together build the quote.}
  \end{quote}
  Erwartetes Ergebnis: Englische Anführungszeichen, weil autonomer Satz.
\end{itemize}

\section{Verifikation: Literaturnachweis}
\begin{itemize}
  \item Buch1 (erstverwendungen)\footcite[vgl.][15]{KantKdrV1974}
  \item ders. ebda.\footcite[vgl.][15]{KantKdrV1974}
  \item ders. a.a.O.\footcite[vgl.][23]{KantKdrV1974}
  \item Buch2 (erstverwendung)\footcite[vgl.][15]{KantKdU1974}
  \item Buch1 (wiederverwendung)\footcite[vgl.][15]{KantKdrV1974}
\end{itemize}

\chapter{Snippet Demo}
\begin{quote}\itshape
In diesem Kapitel erläutern wir \ldots
\end{quote}

\section{Snippet}
% This file originally comes from 'lrt4cs' [(c) 2020 Karsten Reincke,
% https://www.fodina.de/lrt4cs] that is distributed under the terms
% of CC-BY-3.0-DE (= https://creativecommons.org/licenses/by/3.0/)

%% use all entries of the bibliography

\subsection{snippet section 1}
\subsection{snippet section 2}
\subsection{snippet section 3}




\input{bib/ncl.abbrevs-de.tex}
% This file originally comes from 'lrt4cs' [(c) 2020 Karsten Reincke,
% https://www.fodina.de/lrt4cs] that is distributed under the terms
% of CC-BY-3.0-DE (= https://creativecommons.org/licenses/by/3.0/)

\abbr[afda]{AfdA}{Anzeiger für deutsches Altertum}
\abbr[zfda]{ZfdA}{Zeitschrift für deutsches Altertum und deutsche Literatur [ISSN: 00442518]}
\abbr[zfaw]{}{Zeitschrift für Allgemeine Wissenschaftstheorie / Journal for General Philosophy of Science [ISSN: 0044-2216]}

\printnomenclature
\printbibliography

\end{document}
